\documentclass{sig-alternate-2013}

\setlength{\paperheight}{11in}
\setlength{\paperwidth}{8.5in}

\usepackage[utf8]{inputenc}

\usepackage{amsmath,amssymb}
\usepackage{amsthmnoproof}
\usepackage{amsrefs}
\usepackage[usenames,dvipsnames]{color}
\usepackage{stmaryrd}
\usepackage{enumerate}
\usepackage[algoruled,vlined,english,linesnumbered]{algorithm2e}
\usepackage[pdfpagelabels,colorlinks=true,citecolor=blue]{hyperref}
\usepackage{comment}
\usepackage{multirow}
\usepackage{tikz}

\newcommand{\noopsort}[1]{}
\DeclareMathOperator{\NP}{NP}
\DeclareMathOperator{\HP}{HP}
\DeclareMathOperator{\PP}{PP}
\DeclareMathOperator{\Hom}{Hom}
\DeclareMathOperator{\End}{End}
\DeclareMathOperator{\GL}{GL}
\DeclareMathOperator{\val}{val}
\DeclareMathOperator{\pr}{pr}
\DeclareMathOperator{\tr}{Tr}
\DeclareMathOperator{\com}{Com}
\DeclareMathOperator{\Grass}{Grass}
\DeclareMathOperator{\Lat}{Lat}
\DeclareMathOperator{\round}{round}
\DeclareMathOperator{\rank}{rank}

\newcommand{\N}{\mathbb N}
\newcommand{\Z}{\mathbb Z}
\newcommand{\Zp}{\Z_p}
\newcommand{\Q}{\mathbb Q}
\newcommand{\Qp}{\Q_p}
\newcommand{\Fp}{\mathbb{F}_p}
\newcommand{\R}{\mathbb R}
\renewcommand{\O}{\mathcal O}
\newcommand{\OK}{\mathcal{O}_K}
\newcommand{\XX}{\mathbf X}
\newcommand{\trans}{{}^{\text t}}
\newcommand{\T}{\mathcal{T}}

\renewcommand{\prec}{\text{\rm prec}}

\newcommand{\id}{\textrm{id}}
\newcommand{\Epi}{\textrm{Epi}}
\renewcommand{\c}{\text{\rm c}}

\newcommand{\detp}{\det{'}}
\newcommand{\low}{\text{\rm low}}
\newcommand{\up}{\text{\rm up}}
\newcommand{\DI}{\text{\rm DI}}
\newcommand{\II}{\text{\rm II}}
\DeclareMathOperator{\charpoly}{char}
\newcommand{\charp}{\charpoly'}

\newcommand{\lb}{\ensuremath{\llbracket}}
\newcommand{\rb}{\ensuremath{\rrbracket}}
\newcommand{\lp}{(\!(}
\newcommand{\rp}{)\!)}
\newcommand{\col}{\: : \:}

\def\todo#1{\ \!\!{\color{red} #1}}
\definecolor{purple}{rgb}{0.6,0,0.6}
\def\todofor#1#2{\ \!\!{\color{purple} {\bf #1}: #2}}

\def\binom#1#2{\Big(\begin{array}{cc} #1 \\ #2 \end{array}\Big)}


\permission{%
}



\begin{document}

\newtheorem{theo}{Theorem}[section]
\newtheorem{lem}[theo]{Lemma}
\newtheorem{prop}[theo]{Proposition}
\newtheorem{cor}[theo]{Corollary}
\newtheorem{quest}[theo]{Question}
\newtheorem{conj}[theo]{Conjecture}
\theoremstyle{definition}
\newtheorem{rem}[theo]{Remark}
\newtheorem{ex}[theo]{Example}
\newtheorem{deftn}[theo]{Definition}

\title{Multiplication, division and factorization\\of p-adic polynomials}

\numberofauthors{3}
\author{
\alignauthor Xavier Caruso\\
  \affaddr{Universit\'e Rennes 1}\\
  \affaddr{\textsf{xavier.caruso@normalesup.org}}
\alignauthor David Roe \\
  \affaddr{University of British Columbia}\\
  \affaddr{\textsf{roed.math@gmail.com}}
\alignauthor Tristan Vaccon\\
  \affaddr{Universit\'e Rennes 1}\\
  \affaddr{\textsf{tristan.vaccon@univ-rennes1.fr}}
}

\maketitle

\begin{abstract}
\end{abstract}

\category{I.1.2}{Computing Methodologies}{Symbolic and Algebraic Manipulation -- \emph{Algebraic Algorithms}}
\terms{Algorithms, Theory}
\keywords{}

%\vspace{1mm}
% \noindent
% {\bf Categories and Subject Descriptors:} \\
%\noindent I.1.2 [{\bf Computing Methodologies}]:{~} Symbolic and Algebraic
%  Manipulation -- \emph{Algebraic Algorithms}
%
% \vspace{1mm}
% \noindent
% {\bf General Terms:} Algorithms, Theory
%
% \vspace{1mm}
% \noindent
% {\bf Keywords:} $p$-adic precision, linear algebra, ultrametric analysis
%\medskip

\section{Introduction}

\section{Modular multiplication}

\section{Euclidean division}

\section{Slope factorization}

\end{document}
