\documentclass[sigconf]{acmart}

\setlength{\paperheight}{11in}
\setlength{\paperwidth}{8.5in}

\usepackage[utf8]{inputenc}

\hyphenation{regarding}

\usepackage{amsmath,amssymb}
%\usepackage{amsthmnoproof}
\usepackage{amsthm}
\usepackage{mathrsfs}
%\let\bibsection\relax
%\usepackage{amsrefs}
%\usepackage[usenames,dvipsnames]{color}
\usepackage{stmaryrd}
\usepackage{enumerate}
%\usepackage[algoruled,vlined,english,linesnumbered]{algorithm2e}
%\usepackage[pdfpagelabels,colorlinks=true,citecolor=blue]{hyperref}
\usepackage{comment}
\usepackage{multirow}
\usepackage{xspace}
%\usepackage{tikz}

\newcommand{\noopsort}[1]{}
\DeclareMathOperator{\NP}{NP}
\DeclareMathOperator{\HP}{HP}
\DeclareMathOperator{\PP}{PP}
\DeclareMathOperator{\Hom}{Hom}
\DeclareMathOperator{\End}{End}
\DeclareMathOperator{\GL}{GL}
\DeclareMathOperator{\val}{val}
\DeclareMathOperator{\pr}{pr}
\DeclareMathOperator{\tr}{Tr}
\DeclareMathOperator{\adj}{Adj}
\DeclareMathOperator{\Grass}{Grass}
\DeclareMathOperator{\Lat}{Lat}
\DeclareMathOperator{\round}{round}
\DeclareMathOperator{\rank}{rank}
\DeclareMathOperator{\lcm}{lcm}
\DeclareMathOperator{\rec}{rec}
\DeclareMathOperator{\cond}{cond}
\DeclareMathOperator{\disc}{Disc}
\DeclareMathOperator{\row}{row}
\DeclareMathOperator{\col}{col}

\newcommand{\N}{\mathbb N}
\newcommand{\Z}{\mathbb Z}
\newcommand{\Zp}{\Z_p}
\newcommand{\Q}{\mathbb Q}
\newcommand{\Qp}{\Q_p}
\newcommand{\Fp}{\mathbb{F}_p}
\newcommand{\Fq}{\mathbb{F}_q}
\newcommand{\R}{\mathbb R}
\newcommand{\OK}{\mathcal{O}_K}

\newcommand{\ZpL}{\texttt{ZpL}\xspace}
\newcommand{\ZpLCA}{\texttt{ZpLCA}\xspace}
\newcommand{\ZpLCR}{\texttt{ZpLCR}\xspace}
\newcommand{\ZpLF}{\texttt{ZpLF}\xspace}

\newcommand{\famN}{\mathcal{N}}

\newcommand{\llb}{[\mkern-2.5mu[}
\newcommand{\rrb}{]\mkern-2.5mu]}
\newcommand{\llp}{(\mkern-2.5mu(}
\newcommand{\rrp}{)\mkern-2.5mu)}

\newcommand{\calU}{\mathcal{U}}

\newcommand{\softO}{O\tilde{~}}

\newcommand{\inv}{\text{\rm inv}}
\newcommand{\app}{\text{\rm app}}

\def\todo#1{\ \!\!{\color{red} #1}}
\definecolor{purple}{rgb}{0.6,0,0.6}
\def\todofor#1#2{\ \!\!{\color{purple} {\bf #1}: #2}}

\newcommand{\done}[1]{\textcolor{blue}{#1}}
\newcommand{\tdo}[1]{\textcolor{red}{#1}}
\definecolor{answer}{rgb}{0,0.5,0.2}
\newcommand{\xavier}[1]{\textcolor{answer}{{\bf Xavier:} #1}}
\newcommand{\tristan}[1]{\textcolor{answer}{{\bf Tristan:} #1}}
\newcommand{\david}[1]{\textcolor{answer}{{\bf David:} #1}}

\def\binom#1#2{\Big(\begin{array}{cc} #1 \\ #2 \end{array}\Big)}

\clubpenalty=10000
\widowpenalty = 10000

\newtheorem{theo}{Theorem}[section]
\newtheorem{lem}[theo]{Lemma}
\newtheorem{prop}[theo]{Proposition}
\newtheorem{cor}[theo]{Corollary}
\newtheorem{quest}[theo]{Question}
\newtheorem{conj}[theo]{Conjecture}
\theoremstyle{definition}
\newtheorem{rem}[theo]{Remark}
\newtheorem{ex}[theo]{Example}
\newtheorem{deftn}[theo]{Definition}

\fancyhead{}

\begin{document}

\title{\ZpL: a p-adic precision package}

\author{Xavier Caruso}
  \affiliation{Universit\'e Rennes 1; \\
  \institution{IRMAR}
  \city{Rennes, France}
  \postcode{35042}
}
\email{xavier.caruso@normalesup.org}
\author{David Roe}
  \affiliation{University of Pittsburg; \\
  \institution{Department of Mathematics}
  \city{Pittsburgh, PA, USA}
  \postcode{15260}
}
\email{roed@pitt.edu}
\author{Tristan Vaccon}
  \affiliation{Universit\'e de Limoges; \\
  \institution{CNRS, XLIM UMR 7252}
  \city{Limoges, France}  
  \postcode{87060}  
  }
\email{tristan.vaccon@unilim.fr}

\ccsdesc[500]{Computing methodologies~Algebraic algorithms}

\keywords{Algorithms, $p$-adic precision, characteristic polynomial,
eigenvalue}

\begin{abstract}
\end{abstract}

\maketitle

\section{Introduction}

\begin{itemize}
\item Interval arithmetic, floating point arithmetic
\item short introduction of the precision lemma and lattices
\item goal: tracking precision using this theory
\item short presentation of the package
\item plan of the article
\end{itemize}

\section{Examples}

\subsection{Elementary arithmetic}

\begin{itemize}
\item Compare $px$ and $x + x + \cdots +x$ ($p$ times)
\item Compare $x^p$ and $x \cdot x \cdots x$ ($p$ times)
\item $x$ and $y$ are given at different precision, compute $u = x+y$,
$v = x-y$ and $u+v$
\end{itemize}

\subsection{Linear algebra}

\begin{itemize}
\item determinant
\item characteristic polynomial
\end{itemize}

\subsection{Commutative algebra}

\begin{itemize}
\item gcd of polynomials
\end{itemize}

\section{Theory}

\subsection{The precision Lemma}

\subsection{The way the precision is tracked with \ZpL}

We follow the precision lattice.
Closely related to automatic differentiation.

\begin{itemize}
\item \ZpLCR, \ZpLCA: we use a cap to ensure that we have a lattice at
each step.
\item \ZpLF: we work with $\Zp$-submodules of possible positive codimension
and represent them using matrices of floating point $p$-adics.
\end{itemize}

\subsection{The notion of working precision}

\begin{itemize}
\item what is it?
\item how can we compute it? (Answer: Hermite reduction of the transpose)
\item how can we estimate it?
\end{itemize}

\subsection{Correctness}

\begin{itemize}
\item prove that \ZpLCA and \ZpLCR are correct under some assumptions
(to be determined)
\item explain that \ZpLF is not meant to provide proved results
\end{itemize}

\subsection{Complexity}

\begin{tabular}{l|c|c|}
& \ZpLCA/\ZpLCR & \ZpLF \\
\hline
Creation of a variable & $O(n)$ & $O(n_0)$ \\
\hline
Deletion of a variable & $O(m^2)$ & $-$ \\
\hline
\end{tabular}

discussion on temporal locality

\section{Presentation of the package}

\begin{itemize}
\item How to access to the precision
\item Diffused digits
\item Historic of variables creation/deletion, etc...
\item Other features?
\end{itemize}

\subsection{More examples}

\begin{itemize}
\item SOMOS sequence
\item many multiplications of matrices
\item Dodgson algorithm
\item subresultants
\item Grobner basis
\item $p$-adic differential equations
\end{itemize}

\section{Conclusion}

\begin{itemize}
\item In which case should we use this package? Mainly for exploration
\item example of $p$-adic differential equations (especially for $p=2$)
\end{itemize}

\end{document}

