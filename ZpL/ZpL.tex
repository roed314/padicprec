\documentclass[sigconf]{acmart}

\setlength{\paperheight}{11in}
\setlength{\paperwidth}{8.5in}

\usepackage[utf8]{inputenc}

\hyphenation{regarding}

\usepackage{amsmath,amssymb}
%\usepackage{amsthmnoproof}
\usepackage{amsthm}
\usepackage{mathrsfs}
%\let\bibsection\relax
%\usepackage{amsrefs}
%\usepackage[usenames,dvipsnames]{color}
\usepackage{stmaryrd}
\usepackage{enumerate}
%\usepackage[algoruled,vlined,english,linesnumbered]{algorithm2e}
%\usepackage[pdfpagelabels,colorlinks=true,citecolor=blue]{hyperref}
\usepackage{comment}
\usepackage{multirow}
\usepackage{xspace}
\usepackage{tabularx,multicol}
%\usepackage{tikz}

\newcommand{\noopsort}[1]{}
\DeclareMathOperator{\NP}{NP}
\DeclareMathOperator{\HP}{HP}
\DeclareMathOperator{\PP}{PP}
\DeclareMathOperator{\Hom}{Hom}
\DeclareMathOperator{\End}{End}
\DeclareMathOperator{\GL}{GL}
\DeclareMathOperator{\val}{val}
\DeclareMathOperator{\pr}{pr}
\DeclareMathOperator{\tr}{Tr}
\DeclareMathOperator{\adj}{Adj}
\DeclareMathOperator{\Grass}{Grass}
\DeclareMathOperator{\Lat}{Lat}
\DeclareMathOperator{\round}{round}
\DeclareMathOperator{\rank}{rank}
\DeclareMathOperator{\lcm}{lcm}
\DeclareMathOperator{\rec}{rec}
\DeclareMathOperator{\cond}{cond}
\DeclareMathOperator{\disc}{Disc}
\DeclareMathOperator{\row}{row}
\DeclareMathOperator{\col}{col}

\newcommand{\N}{\mathbb N}
\newcommand{\Z}{\mathbb Z}
\newcommand{\Zp}{\Z_p}
\newcommand{\Q}{\mathbb Q}
\newcommand{\Qp}{\Q_p}
\newcommand{\Fp}{\mathbb{F}_p}
\newcommand{\Fq}{\mathbb{F}_q}
\newcommand{\R}{\mathbb R}
\newcommand{\OK}{\mathcal{O}_K}

\newcommand{\ZpL}{\texttt{ZpL}\xspace}
\newcommand{\ZpLCA}{\texttt{ZpLCA}\xspace}
\newcommand{\ZpLCR}{\texttt{ZpLCR}\xspace}
\newcommand{\ZpLF}{\texttt{ZpLF}\xspace}

\newcommand{\famN}{\mathcal{N}}

\newcommand{\llb}{[\mkern-2.5mu[}
\newcommand{\rrb}{]\mkern-2.5mu]}
\newcommand{\llp}{(\mkern-2.5mu(}
\newcommand{\rrp}{)\mkern-2.5mu)}

\newcommand{\calU}{\mathcal{U}}

\newcommand{\softO}{O\tilde{~}}

\newcommand{\inv}{\text{\rm inv}}
\newcommand{\app}{\text{\rm app}}

\def\todo#1{\ \!\!{\color{red} #1}}
\definecolor{purple}{rgb}{0.6,0,0.6}
\def\todofor#1#2{\ \!\!{\color{purple} {\bf #1}: #2}}

\newcommand{\done}[1]{\textcolor{blue}{#1}}
\newcommand{\tdo}[1]{\textcolor{red}{#1}}
\definecolor{answer}{rgb}{0,0.5,0.2}
\newcommand{\xavier}[1]{\textcolor{answer}{{\bf Xavier:} #1}}
\newcommand{\tristan}[1]{\textcolor{answer}{{\bf Tristan:} #1}}
\newcommand{\david}[1]{\textcolor{answer}{{\bf David:} #1}}

\def\binom#1#2{\Big(\begin{array}{cc} #1 \\ #2 \end{array}\Big)}

\clubpenalty=10000
\widowpenalty = 10000

\newtheorem{theo}{Theorem}[section]
\newtheorem{lem}[theo]{Lemma}
\newtheorem{prop}[theo]{Proposition}
\newtheorem{cor}[theo]{Corollary}
\newtheorem{quest}[theo]{Question}
\newtheorem{conj}[theo]{Conjecture}
\theoremstyle{definition}
\newtheorem{rem}[theo]{Remark}
\newtheorem{ex}[theo]{Example}
\newtheorem{deftn}[theo]{Definition}

\fancyhead{}

\begin{document}

\title{\ZpL: a p-adic precision package}

\author{Xavier Caruso}
  \affiliation{Universit\'e Rennes 1; \\
  \institution{IRMAR}
  \city{Rennes, France}
  \postcode{35042}
}
\email{xavier.caruso@normalesup.org}
\author{David Roe}
  \affiliation{University of Pittsburg; \\
  \institution{Department of Mathematics}
  \city{Pittsburgh, PA, USA}
  \postcode{15260}
}
\email{roed@pitt.edu}
\author{Tristan Vaccon}
  \affiliation{Universit\'e de Limoges; \\
  \institution{CNRS, XLIM UMR 7252}
  \city{Limoges, France}  
  \postcode{87060}  
  }
\email{tristan.vaccon@unilim.fr}

\ccsdesc[500]{Computing methodologies~Algebraic algorithms}

\keywords{Algorithms, $p$-adic precision, characteristic polynomial,
eigenvalue}

\begin{abstract}
\end{abstract}

\maketitle

\section{Introduction}

\begin{itemize}
\item Interval arithmetic, floating point arithmetic
\item short introduction of the precision lemma and lattices
\item goal: tracking precision using this theory
\item short presentation of the package
\item plan of the article
\end{itemize}

\section{Examples}

\subsection{Elementary arithmetic}

\begin{tabular}{|l|l|}
\verb?R = Zp(3,5)? & \verb?R = ZpL(3,5)? \\
\verb?x = R(143,5); x? & \verb?x = R(143,5); x? \\
\hfill\verb?...12022? & \hfill\verb?...12022? \\
\verb?3*x? & \verb?3*x? \\
\hfill\verb?...120220? & \hfill\verb?...120220? \\
\verb?x + x + x? & \verb?x + x + x? \\
\hfill\verb?...20220? & \hfill\verb?...120220? \\
\verb?x^3? & \verb?x^3? \\
\hfill\verb?...020222? & \hfill\verb?...020222? \\
\verb?x * x * x? & \verb?x * x * x? \\
\hfill\verb?...20222? & \hfill\verb?...020222? \\
\end{tabular}

\begin{tabular}{|l|l|}
\verb?R = Zp(2,10)? & \verb?R = ZpL(2,10)? \\
\verb?x = R(987,10); x? & \verb?x = R(987,10); x? \\
\hfill\verb?...1111011011? & \hfill\verb?...1111011011? \\
\verb?y = R(21,5); y? & \verb?y = R(21,5); y? \\
\hfill\verb?...10101? & \hfill\verb?...10101? \\
\verb?u = x + y; u? & \verb?u = x + y; u? \\
\hfill\verb?...10000? & \hfill\verb?...10000? \\
\verb?v = x - y; v? & \verb?v = x - y; v? \\
\hfill\verb?...00110? & \hfill\verb?...00110? \\
\verb?u + v? & \verb?u + v? \\
\hfill\verb?...10110? & \hfill\verb?...11110110110? \\
\end{tabular}

\subsection{Linear algebra}

\begin{itemize}
\item determinant
\item characteristic polynomial
\end{itemize}

\subsection{Commutative algebra}

\begin{itemize}
\item gcd of polynomials
\end{itemize}

\section{Theory}

\subsection{The precision Lemma}

\subsection{The way the precision is tracked with \ZpL}

We follow the precision lattice.
Closely related to automatic differentiation.

\begin{itemize}
\item \ZpLCR, \ZpLCA: we use a cap to ensure that we have a lattice at
each step.
\item \ZpLF: we work with $\Zp$-submodules of possible positive codimension
and represent them using matrices of floating point $p$-adics.
\end{itemize}

\subsection{The notion of working precision}

\begin{itemize}
\item what is it?
\item how can we compute it? (Answer: Hermite reduction of the transpose)
\item how can we estimate it?
\end{itemize}

\subsection{Correctness}

\begin{itemize}
\item prove that \ZpLCA and \ZpLCR are correct under some assumptions
(to be determined)
\item explain that \ZpLF is not meant to provide proved results
\end{itemize}

\subsection{Complexity}

\begin{tabular}{l|c|c|}
& \ZpLCA/\ZpLCR & \ZpLF \\
\hline
Creation of a variable & $O(n)$ & $O(n_0)$ \\
\hline
Deletion of a variable & $O(m^2)$ & $-$ \\
\hline
\end{tabular}

discussion on temporal locality

\section{Presentation of the package}

\begin{itemize}
\item Discuss coercion/conversion: \texttt{Zp} coerces to \ZpL but no coersion
in the other direction
\item How to access to the precision
\item Diffused digits
\item Historic of variables creation/deletion, etc...
\item Other features?
\end{itemize}

\subsection{More examples}

\begin{tabular}{|l|l|}
\verb?R = Zp(2)? & \verb?R = ZpL(2)? \\
\multicolumn{2}{|l|}{\tt x, y, z, t = R(1,15), R(1,15), R(1,15), R(3,15)} \\
\multicolumn{2}{|l|}{\tt for \_ in range(20):} \\
\multicolumn{2}{|l|}{\tt \ \ \ \ x, y, z, t = y, z, t, (y*t + z*z)/x} \\
\multicolumn{2}{|l|}{\tt \ \ \ \ print t} \\
\hfill\verb?...000000000000100? & \hfill\verb?...000000000000100? \\
\hfill\verb?...000000000001101? & \hfill\verb?...000000000001101? \\
\hfill\verb?...000000000110111? & \hfill\verb?...000000000110111? \\
\hfill\verb?...101010111010111? & \hfill\verb?...101010111010111? \\
\hfill\verb?...1100111101111? & \hfill\verb?...101100111101111? \\
\hfill\verb?...1110000010010? & \hfill\verb?...111110000010010? \\
\hfill\verb?...0001000111001? & \hfill\verb?...100001000111001? \\
\hfill\verb?...0000011111101? & \hfill\verb?...100000011111101? \\
\hfill\verb?...1000000110101? & \hfill\verb?...001000000110101? \\
\hfill\verb?...101101010011? & \hfill\verb?...010101101010011? \\
\hfill\verb?...110000000000? & \hfill\verb?...001110000000000? \\
\hfill\verb?...000101011101? & \hfill\verb?...111000101011101? \\
\hfill\verb?...001001101011? & \hfill\verb?...111001001101011? \\
\hfill\verb?...000011110011? & \hfill\verb?...111000011110011? \\
\hfill\verb?...11? & \hfill\verb?...0000000111? \\
\hfill\verb?...10? & \hfill\verb?...1010001110? \\
\hfill\verb?...01? & \hfill\verb?...1110110001? \\
\hfill\verb?...01? & \hfill\verb?...011110111001? \\
\hfill\verb?...01? & \hfill\verb?...100101101001001? \\
\hfill\verb?...1? & \hfill\verb?...11111000011? \\
\end{tabular}

\begin{itemize}
\item SOMOS sequence
\item many multiplications of matrices
\item Dodgson algorithm
\item subresultants
\item Grobner basis
\item $p$-adic differential equations
\end{itemize}

\subsubsection{$p$-adic differential equations}


In \cite{LV16}, the behaviour of the precision when solving
$p$-adic differential equations with separation of 
variables has been studied.
The authors have investigated the gap when that appears
when applying a Newton-method solver between
the theoretic loss in precision and the 
actual loss in precision for a naive implemenationin \verb?Zp(p)?.
We can reach this theoretical loss in precision using \verb?ZpL?.
We use a generic \verb?Newton_Iteration_Solver(g,h,N)?
that applies \verb?N? steps of the Newton method for 
$y'=g \times h(y)$ as described in \citep{LV16}. 

\begin{tabular}{|l|l|}
\verb?R = Qp(2,250)? & \verb?R = QpL(2,250)? \\
\verb?S = Zp(2,250)? & \verb?S = ZpL(2,250)? \\
\multicolumn{2}{|l|}{\tt A.<x> = PowerSeriesRing(R,32) } \\
\multicolumn{2}{|l|}{\tt B.<x> = PowerSeriesRing(S,32)) } \\
\multicolumn{2}{|l|}{\tt h,y =1+x+x\^{}3 , x+x\^{}2*A(B.random\_element()) } \\
\multicolumn{2}{|l|}{\tt g=y.derivative()/h(y) } \\
\multicolumn{2}{|l|}{\tt s=Newton\_Iteration\_Solver(g,h,5)} \\
\multicolumn{2}{|l|}{\tt print 250-(s[31].lift()-y[31].lift()).valuation(2)} \\
\hfill\verb?9? & \hfill\verb?4? \\
\end{tabular}
The theorical loss in precision for the coefficient of $t^{31}$
is $\lfloor log_2 (31)\rfloor=4$ and is then reached, whereas the
loss in precision for a naive implementation is clearly more severe.


\section{Conclusion}

\begin{itemize}
\item In which case should we use this package? Mainly for exploration
\item example of $p$-adic differential equations (especially for $p=2$)
\end{itemize}

\subsection{Exploration}

The purpose of this package is that of exploration
of the $p$-adic precision.
We conclude with a last example coming
from the field of the computation of
isogenies between elliptic curves over finite fields.
Techniques to compute these isogenies
through the solving of $p$-adic differential equations
have a long history, and a first
analysis of the behaviour of precision using
lattices has been investigated in \cite{LV16}.

In that article, the authors were interested in
solving $y'^2=g \times h(y)$ with
$g,h,y \in \mathbb{Z}_p \llbracket x \rrbracket$
such that $g(0)=h(0)=1$ and $y(0)=0.$

The main result was that the intrinsic loss in precision
when computing the coefficient $x^n$ of $y$
from $g$ and $h$ was in $\log_p(n)$
even though a naive analysis of
the Newton method for solving the equation
yield a loss in $\log_p(n)^2.$

However, this does not encompass the case of 
elliptic curves defined and isogeneous 
over $\mathbb{Q},$ which can yield a differential 
equation with a solution $y$ whose coefficients
have negative valuation.

For instance, the curves 
$E: \: y^2=x^3+1$
and $E': \: y^2=+ (m^2-1)x^2/4 + m^6$ are isogeneous
over $\mathbb{Q}$ if  $m$ is an odd integer.
We take $m=67.$ The corresponding isogeny has coefficients with possibly
negative $3$-adic valuation.

We can solve numerically 
the corresponding differential equation
using the algorithm of \cite{Lercier-Sirvent:08}
over $\mathbb{Q}_3$ with lattice precision.
Any generic implementation of the Newton
method of \cite{Lercier-Sirvent:08} is enough.

\begin{tabular}{|l|}
\verb?    R = QpL(3)? \\
\verb?    A.<x> = PolynomialRing(R)?\\
\verb?    m=67?\\
\verb?    g=1+(1/4)*x^2+x^6?\\
\verb?    h=1+((m^2)/4)*x^2+m^6* x^6?\\
\verb?    y=Lercier_Sirvent(g,h,64)?\\
\verb?    y[31]?\\
\verb?(467995903278061284942799/3^14 + O(3^36))*x^31?\\
\end{tabular}

We can then see that up to $63,$ the behaviour
of the loss in relative precision is the same
as for the coefficient in $x^{31}$: \textbf{no loss
occured}.
Thus, those simple manipulations
hint at a different behaviour for the loss 
in precision in at least some special cases of
of isogenies defined over $\mathbb{Q}$
but not necessarily over $\mathbb{Z}_p.$ 

\bibliographystyle{plain}
\bibliography{ZpL}


\end{document}

