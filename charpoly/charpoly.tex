\documentclass{sig-alternate-05-2015}

\setlength{\paperheight}{11in}
\setlength{\paperwidth}{8.5in}

\usepackage[utf8]{inputenc}

\hyphenation{regarding}

\usepackage{amsmath,amssymb}
\usepackage{amsthmnoproof}
\usepackage{amsrefs}
\usepackage[usenames,dvipsnames]{color}
\usepackage{stmaryrd}
\usepackage{enumerate}
\usepackage[algoruled,vlined,english,linesnumbered]{algorithm2e}
\usepackage[pdfpagelabels,colorlinks=true,citecolor=blue]{hyperref}
\usepackage{comment}
\usepackage{multirow}
\usepackage{tikz}

\newcommand{\noopsort}[1]{}
\DeclareMathOperator{\NP}{NP}
\DeclareMathOperator{\HP}{HP}
\DeclareMathOperator{\PP}{PP}
\DeclareMathOperator{\Hom}{Hom}
\DeclareMathOperator{\End}{End}
\DeclareMathOperator{\GL}{GL}
\DeclareMathOperator{\val}{val}
\DeclareMathOperator{\pr}{pr}
\DeclareMathOperator{\tr}{Tr}
\DeclareMathOperator{\com}{Com}
\DeclareMathOperator{\Grass}{Grass}
\DeclareMathOperator{\Lat}{Lat}
\DeclareMathOperator{\round}{round}
\DeclareMathOperator{\rank}{rank}
\DeclareMathOperator{\lcm}{lcm}

\newcommand{\N}{\mathbb N}
\newcommand{\Z}{\mathbb Z}
\newcommand{\Zp}{\Z_p}
\newcommand{\Q}{\mathbb Q}
\newcommand{\Qp}{\Q_p}
\newcommand{\Fp}{\mathbb{F}_p}
\newcommand{\R}{\mathbb R}

\newcommand{\softO}{O\tilde{~}}

\def\todo#1{\ \!\!{\color{red} #1}}
\definecolor{purple}{rgb}{0.6,0,0.6}
\def\todofor#1#2{\ \!\!{\color{purple} {\bf #1}: #2}}

\def\binom#1#2{\Big(\begin{array}{cc} #1 \\ #2 \end{array}\Big)}

\CopyrightYear{2016}
\setcopyright{acmcopyright}
\conferenceinfo{ISSAC '16,}{July 19-22, 2016, Waterloo, ON, Canada}
\isbn{978-1-4503-4380-0/16/07}\acmPrice{\$15.00}
\doi{http://dx.doi.org/10.1145/2930889.2930898}

\permission{Publication rights licensed to ACM. ACM acknowledges that this contribution was authored or co-authored by an employee, contractor or affiliate of a national government. As such, the Government retains a nonexclusive, royalty-free right to publish or reproduce this article, or to allow others to do so, for Government purposes only.}

\begin{document}

\newtheorem{theo}{Theorem}[section]
\newtheorem{lem}[theo]{Lemma}
\newtheorem{prop}[theo]{Proposition}
\newtheorem{cor}[theo]{Corollary}
\newtheorem{quest}[theo]{Question}
\newtheorem{conj}[theo]{Conjecture}
\theoremstyle{definition}
\newtheorem{rem}[theo]{Remark}
\newtheorem{ex}[theo]{Example}
\newtheorem{deftn}[theo]{Definition}

\title{Characteristic polynomial of p-adic matrices}

\numberofauthors{3}
\author{
\alignauthor Xavier Caruso\\
  \affaddr{Universit\'e Rennes 1}\\
  \affaddr{\textsf{xavier.caruso@normalesup.org}}
\alignauthor David Roe\\
  \affaddr{Pittsburg University}\\
  \affaddr{\textsf{roed@pitt.edu}}
\alignauthor Tristan Vaccon\\
  \affaddr{Universit\'e de Limoges}\\
  \affaddr{\textsf{tristan.vaccon@unilim.fr}}
}

\maketitle

\begin{abstract}
\end{abstract}

\begin{CCSXML}
<ccs2012>
<concept>
<concept_id>10010147.10010148.10010149.10010150</concept_id>
<concept_desc>Computing methodologies~Algebraic algorithms</concept_desc>
<concept_significance>500</concept_significance>
</concept>
</ccs2012>
\end{CCSXML}

\vspace{-1mm}
\ccsdesc[500]{Computing methodologies~Algebraic algorithms}
\printccsdesc

\vspace{-1.5mm}
\keywords{Algorithms, $p$-adic precision, Newton polygon, factorization}

%\vspace{1mm}
% \noindent
% {\bf Categories and Subject Descripto\RS:} \\
%\noindent I.1.2 [{\bf Computing Methodologies}]:{~} Symbolic and Algebraic
%  Manipulation -- \emph{Algebraic Algorithms}
%
% \vspace{1mm}
% \noindent
% {\bf General Terms:} Algorithms, Theory
%
% \vspace{1mm}
% \noindent
% {\bf Keywords:} $p$-adic precision, linear algebra, ultrametric analysis
%\medskip

\section{Introduction}

\todo{Example of $M$ and $M + 1$}

\section{Theoretical study}

\subsection{The theory of p-adic precision}

\todo{Recall the Lemma (with integral version)}

\subsection{Computation of the differential}

\todo{Formula with comatrix.}

\subsection{Properties of the image of $d \chi$}

\todo{Surjectivity. Stability under multiplication by $X$ mod $\chi$.}

\todo{Image modulo $p$ and elementary divisors. 
Consequences: (1)~computation in $\softO(n^\omega)$ and (2)~gain criterium.}

\section{Computation of $\chi$ and $d\chi$}

\subsection{Hessenberg form}

\todo{Describe algorithm (with matrices and endormophisms?). 
Remark about precision.}

\subsection{Computation of the inverse}

\todo{Describe algorithm with SNF. Remark about precision.
Application to the computation of $d\chi$}

\subsection{Jagged precision}

\todo{Computation of optimal jagged precision.}

\section{The case of diagonalizable matrices}

\subsection{Legendre precision}

\todo{Describe Legendre precision and show that it is optimal for
a diagonalizable matrix over $\Zp$.}

\subsection{Algorithm}

\end{document}
