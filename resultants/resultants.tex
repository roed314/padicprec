\documentclass{sig-alternate}
%\documentclass{lms}
\usepackage[utf8]{inputenc}

\usepackage{amsmath,amssymb}
%\usepackage{amsrefs}
\usepackage[usenames,dvipsnames]{color}
\usepackage{stmaryrd}
\usepackage{enumerate}
%\usepackage[algoruled,vlined,english,linesnumbered]{algorithm2e}
\usepackage[pdfpagelabels,colorlinks=true,citecolor=blue]{hyperref}
\usepackage{comment}
\usepackage{tikz}
\usepackage{bbm}
\usepackage{multirow}
\renewcommand*{\theHsection}{\thesection}

\newcommand{\noopsort}[1]{}
\DeclareMathOperator{\NP}{NP}
\DeclareMathOperator{\GL}{GL}
\DeclareMathOperator{\val}{val}
\DeclareMathOperator{\pr}{pr}
\DeclareMathOperator{\tr}{Tr}
\DeclareMathOperator{\com}{Com}
\DeclareMathOperator{\Grass}{Grass}
\DeclareMathOperator{\Lat}{Lat}
\DeclareMathOperator{\round}{round}

\begin{document}

\newtheorem{theo}{Theorem}[section]
\newtheorem{lem}[theo]{Lemma}
\newtheorem{prop}[theo]{Proposition}
\newtheorem{cor}[theo]{Corollary}
\newtheorem{quest}[theo]{Question}
%\theoremstyle{definition}
\newtheorem{rem}[theo]{Remark}
\newtheorem{ex}[theo]{Example}
\newtheorem{deftn}[theo]{Definition}
\newtheorem{rmk}[theo]{Remark}

\newcommand{\N}{\mathbb N}
\newcommand{\Z}{\mathbb Z}
\newcommand{\Zp}{\Z_p}
\newcommand{\Q}{\mathbb Q}
\newcommand{\Qp}{\Q_p}
\newcommand{\F}{\mathbb F}
\newcommand{\Fp}{\F_p}
\newcommand{\R}{\mathbb R}
\renewcommand{\O}{\mathcal O}
\newcommand{\m}{\mathfrak m}

\newcommand{\M}{\text{\tt M}}

\newcommand{\ring}{\mathfrak A}
\newcommand{\fracring}{\mathfrak K}
\renewcommand{\prec}{\text{\rm prec}}
\renewcommand{\val}{\text{\rm val}}
\newcommand{\id}{\text{\rm id}}
\newcommand{\Res}{\text{\rm Res}}
\newcommand{\lc}{\text{\rm lc}}
\newcommand{\Card}{\text{\rm Card}\:}

\renewcommand{\P}{\mathbb P}
\newcommand{\E}{\mathbb E}
\newcommand{\Var}{\text{\rm Var}}
\newcommand{\Cov}{\text{\rm Cov}}


\newcommand{\lb}{\ensuremath{\llbracket}}
\newcommand{\rb}{\ensuremath{\rrbracket}}
\newcommand{\lp}{(\!(}
\newcommand{\rp}{)\!)}
\newcommand{\col}{\: : \:}

\newcommand{\A}{W}

\def\todo#1{\ \!\!{\color{red} #1}}
\definecolor{purple}{rgb}{0.6,0,0.6}
\def\todofor#1#2{\ \!\!{\color{purple} {\bf #1}: #2}}

\def\binom#1#2{\Big(\begin{array}{cc} #1 \\ #2 \end{array}\Big)}

\title{Resultants and subresultants of p-adic polynomials}

\numberofauthors{1}
\author{
\alignauthor Xavier Caruso\\
  \affaddr{Universit\'e Rennes 1}\\
  \email{\normalsize \textsf{xavier.caruso@normalesup.org}}
}

\maketitle

\begin{abstract}
We address the problem of the stability of the computations of resultants 
and subresultants of polynomials defined over complete discrete valuation 
rings (\emph{e.g.} $\Zp$ or $k[[t]]$ where $k$ is a field). We prove that 
Euclide-like algorithms are highly unstable on average and we explain, in
many cases, how one can stabilize them without sacrifying the complexity.
On the way, we completely determine the distribution of the valuation of
the subresultants of two random monic $p$-adic polynomials having the
same degree.
\end{abstract}

\vspace{1mm}
 \noindent
 {\bf Categories and Subject Descriptors:} \\
\noindent I.1.2 [{\bf Computing Methodologies}]:{~} Symbolic and Algebraic
  Manipulation -- \emph{Algebraic Algorithms}

 \vspace{1mm}
 \noindent
 {\bf General Terms:} Algorithms, Theory

 \vspace{1mm}
 \noindent
 {\bf Keywords:} $p$-adic precision, subresultants, random $p$-adic 
polynomials
\medskip

\section{Introduction}

As wonderfully illustrated by the success of Kedlaya-type counting points 
algorithms \cite{kedlaya}, $p$-adic technics are gaining nowadays more and more 
popularity in computer science, and more specifically in symbolic 
computation. A crucial issue when dealing with $p$-adics is those of 
stability. Indeed, just like real numbers, $p$-adic numbers are by nature 
infinite and thus need to be truncated in order to fit in the memory of a 
computer. The level of truncation is called the \emph{precision}. Usual 
softwares implementing $p$-adics (\emph{e.g.} \textsc{magma} \cite{magma}, 
\textsc{pari} \cite{pari}, \textsc{sage} \cite{sage}) 
generally tracks the precision as follows: an individual precision is 
attached to any $p$-adic variable and this precision is updated after 
each basic arithmetic operation (addition and multiplication). This way 
of tracking precision can be seen as the analogue of the arithmetic 
intervals in the real setting.

In the paper \cite{padicprec}, the authors propose a new framework to control 
$p$-adic precision. The aim of this paper is to illustrate the technics 
of \emph{loc. cit.} on the concrete example of computation of 
\textsc{gcd}s and subresultants of $p$-adic polynomials. There is 
actually a real need to do this due to the combination of two reasons: on 
the one hand, computating \textsc{gcd}s is a very basic operation for 
which it cannot be acceptable to have important instability whereas, on 
the other hand, easy experimentations show that all standard algorithms 
for this task (\emph{e.g.} extended Euclide's algorithm) are \emph{very} 
unstable.
%
\begin{figure}
\renewcommand{\arraystretch}{1.2}
\begin{center}
\begin{tabular}{|c||c|c|}
\hline
\multirow{2}{*}{\hspace{0.2cm}Degree\hspace{0.2cm}\null} & 
\multicolumn{2}{c|}{\hspace{0.3cm}Loss of precision (in number of digits)\hspace{0.3cm}\null}\\
\cline{2-3}
& \hspace{0.3cm}Euclide algorithm\hspace{0.3cm}\null & expected \\
\hline
$5$ & $\phantom{00}6.3$ & $3.1$ \\
$10$ & $\phantom{0}14.3$ & $3.2$ \\
$25$ & $\phantom{0}38.9$ & $3.2$ \\
$50$ & $\phantom{0}79.9$ & $3.2$ \\
$100$ & $160.0$ & $3.2$ \\
\hline
\end{tabular}
\end{center}

\vspace{-0.3cm}

\caption{Average loss of precision when computing the {\sc gcd}
of two random monic polynomial of fixed degree over $\Z_2$.}
\label{fig:precision}

\vspace{-0.3cm}
\end{figure}
%
Figure \ref{fig:precision} illustrates the instability of extended Euclide's 
algorithm when it is called on random inputs which are monic $2$-adic 
polynomials of fixed degree. Looking at the last line, we see that 
extended Euclide's algorithm outputs the B\'ezout coefficients of two 
monic $2$-adic polynomials of degree $100$ with an average loss of 
$161.1$ significant digits by coefficient whereas a stable 
algorithm should only loose $3.2$ digits on average. This 
``theoretical'' loss is computed as the double of the valuation of the 
resultant. Indeed Cramer-like formulae imply that B\'ezout coefficients 
can be computed by performing a unique division by the resultant, 
inducing then only the aforementioned loss of precision. Examining the 
table a bit more, we furthermore observe that the ``practical'' loss of 
precision due to Euclide's algorithm seems to grow linearly with respect 
to the degree of the input polynomials whereas the ``theoretical'' loss 
seems to be independant of it. In other words, the instability of 
Euclide's algorithm is becoming more and more critical when the degree 
of the input increases.

\medskip

\noindent
{\bf Content of the paper.}
The aim of this article is twofold. We first provide in \S 
\ref{sec:unstable} a theoretical study of the instability phenomenon 
described above and give strong evidences that the loss of precision 
grows linearly with respect to the degree of the input polynomials, as we 
observed empirically. In doing so, we determine the distribution of the 
valuation of the subresultants of random monic polynomials over $\Zp$ 
(\emph{cf} Theorem \ref{th:lawVj}). This is an independant result which 
has its own interest.

Our second goal, which is carried out in \S \ref{sec:stable}, is to rub 
out these unexpected losses of precision. Making slight changes to the 
standard subresultants pseudo-remainder sequence algorithm and using in 
an essential way the results of \cite{padicprec}, we manage to design a stable 
algorithm for computing all subresultants of two monic polynomials over 
$\Zp$ (satisfying an additional mild assumption). This basically allows 
to stably compute \textsc{gcd}s assuming that the degree of the 
\textsc{gcd} is known in advance.

\section{The setting}

The aim of this section is to introduce the setting we shall work in 
throughout this paper (which is a bit more general than those considered 
in the introduction).

\subsection{Complete discrete valuation rings}

\noindent
\textbf{Definitions.}
Let $\A$ be a \emph{discrete valuation ring} (DVR for short), that is a 
domain equipped with a map $\val : \A \to \Z \cup \{+\infty\}$ --- the 
so-called \emph{valuation} --- satisfying the four axioms: (1)~$\val(x) 
= +\infty$ iff $x = 0$, (2)~$\val(xy) = \val(x) + \val(y)$ and 
(3)~$\val(x+y) \geq \min(\val(x), \val(y))$, (4)~any element of 
valuation $0$ is invertible. In the sequel, we shall assume that the
valuation is normalized so that it takes the value $1$. We recall that a 
DVR admits a unique maximal ideal $\m$, consisting of elements of 
positive valuation. Furthermore, this ideal is principal and generated 
by any element of valuation $1$. Such an element is called a 
\emph{uniformizer}. Let us fix one of them and denote it by $\pi$. The 
\emph{residue field} of $W$ is the quotient $W/\m = W/\pi W$ and we 
shall denote it by $k$.

The valuation defines a distance $d$ on $\A$ by letting $d(x,y) = 
e^{-\val(x-y)}$ for all $x, y \in \A$. We say that $\A$ is \emph{complete}
if it is complete with respect to the previous distance. Assuming that
$\A$ is complete, any element $x \in \A$ can be written uniquely as a 
convergent series:
\begin{equation}
\label{eq:expandCDVR}
x = x_0 + x_1 \pi + x_2 \pi^2 + \cdots + x_n \pi^n + \cdots
\end{equation}
where the $x_i$'s lie in a fixed set of representatives of classes
modulo $\pi$. Therefore, as an additive group, $W$ is isomorphic to 
the set of sequences $\N \to k$. On the contrary, the multiplicative
structure may vary.
Let $K$ denote the fraction field of $\A$. The valuation $v$ extends 
uniquely to $K$ by letting $\val(\frac x y) = \val(x) - \val(y)$. 
Moreover, it follows from the axiom~(4) that $K$ is obtain from $\A$ by 
inverting $\pi$. Thus, any element of $K$ can be uniquely written as an 
infinite sum $x = \sum_{i=v}^\infty x_i \pi^i$ where $v$ is some 
relative integer and the $x_i$'s are as above. The valuation of $x$ can 
be easily read on this writing: it is the smallest integer $v$ such that 
$x_i \not\equiv 0 \pmod \pi$.

The two main examples of complete discrete valuation rings are the 
ring of $p$-adic integers $\Zp$ and the ring of formal series 
$k[[x]]$ over a field $k$. Their fraction fields are $\Q_p$ and
$k((x))$ respectively.

\medskip

\noindent
\textbf{Precision.}
The memory of a computer being necessarily finite, it cannot be possible 
to find a way to represent exhaustively all elements of $\A$. 
Very often, mimicing what we do for real numbers, we choose to truncate 
the expansion \eqref{eq:expandCDVR} at some finite level. Concretely,
this means that we work with approximations of elements of $\A$ of the
form
$\sum_{i=0}^{N-1} x_i \pi^i + O(\pi^N)$
where $N$ is some integer called the \emph{precision} of $x$ and 
the term $O(\pi^N)$ means that the $x_i$'s with $i \geq N$ are not 
specified. Hence, the notation above does not represent a unique
element of $\A$ but all elements lying in the ball having center
$\sum_{i=0}^{N-1} x_i \pi^i$ and radius $e^{-N v(\pi)}$. In other
words, on a computer, we do not work with elements of $\A$ but with
balls in $\A$ which are more tractable.

\subsection{Subresultants}
\label{subsec:subres}

A first issue when dealing with numerical computations of \textsc{gcd}s 
of polynomials over $\A$ is that the \textsc{gcd} function is not 
continuous: it takes the value $1$ on an open dense subset without being 
constant. This of course annihilate any hope of computing \textsc{gcd}s 
of polynomials when only approximations of them are known. Fortunately, 
there exists a standard way to recover continuity in this context: it 
consists in replacing \textsc{gcd}s by the so-called subresultants which 
are playing an analoguous role. For this reason, in what follows, we will 
exclusively consider the problem of computing subresultants.

\medskip

\noindent
\textbf{Notations.} We refer to \cite[\S 3.3]{cohen} and \cite[\S 4.1]{winkler}
for a nice 
introduction to resultants and subresultants and seek to fix notations
in this paragraph. For now, we work over a 
general ring $\ring$. We denote by $\ring_{<n}[X]$ (resp. $\ring_{\leq 
n}[X]$) the set of polynomials of degree less than $n$ (resp. less than 
or equal to $n$) with coefficients in $\ring$. Both $\ring_{<n}[X]$ and 
$\ring_{\leq n}[X]$ are free $\ring$-modules of finite rank.

Let $A$ and $B$ be two polynomials with coefficients in $\ring$ and 
$d_A$ and $d_B$ be two integer greater than or equal to the degree of 
$A$ and $B$ respectively. Given an integer $j \in \{0, \ldots, \min 
(d_A, d_B)-1\}$, we denote by $\Res_j^{d_A, d_B}(A,B)$ the $j$-th 
\emph{subresultant} of $(A,B)$ ``computing in degree $(d_A, d_B)$'' and 
by $U_j^{d_A, d_B}(A,B)$ and $V_j^{d_A, d_B}(A,B)$ the corresponding 
\emph{cofactors}. In what follows, we will freely drop the exponent 
$(d_A,d_B)$ when $d_A$ and $d_B$ are the degrees of $A$ and $B$ 
respectively.

We recall briefly that $\Res_j^{d_A, d_B}(A,B)$ (resp. $U_j^{d_A, 
d_B}(A,B)$, resp. $V_j^{d_A, d_B}(A,B)$) is a polynomial with 
coefficients in $\ring$ of degree at most $j$ (resp. $d-j-1$, resp. 
$d-j-1$) whose coefficients can be expressed, up to a sign, as minors
of the Sylvester matrix. These polynomials satisfy in addition the
relation:
$$A \cdot U_j^{d_A,d_B}(A,B) + B \cdot V_j^{d_A,d_B}(A,B) = 
\Res^{d_A,d_B}_j(A,B).$$
The $j$-th coefficient of $R_j^{d_A,d_B}(A,B)$ is sometimes called the
$j$-th \emph{scalar subresultant} of $(A,B)$.
When $j=0$, the polynomial $\Res^{d_A,d_B}_0(A,B)$ is a constant which
is nothing but the resultant of $(A,B)$. In the sequel, we shall denote
it simply $\Res^{d_A,d_B}(A,B)$ (thus removing the index $0$).
The relation between subresultants and \textsc{gcd}s is given by the
following well-known theorem.

\begin{theo}
We assume that $\ring$ is a field. Let $A$ and $B$ be two polynomials
with coefficients in $\ring$. Let $j$ be the smallest integer such
that $\Res_j(A,B)$ does not vanish. Then $\Res_j(A,B)$ is a \textsc{gcd}
of $A$ and $B$.
\end{theo}

Since they are defined as determinants, subresultants behave well with 
respect to base change: if $f : \ring \to \ring'$ is a morphism of rings 
and $A$ and $B$ are polynomials over $\ring$ then
$\Res_j^{d_A,d_B}(f(A),f(B)) = f\big(\Res_j^{d_A,d_B}(A,B)\big)$
where $f(A)$ and $f(B)$ denotes the polynomials deduced from $A$
and $B$ respectively by applying $f$ coefficient-wise. This property
is sometimes referred to as the \emph{functoriality} of subresultants.

\medskip

\noindent
\textbf{The subresultant pseudo-remainder sequence.}
When $\ring$ is a domain, there exists a standard nice Euclide-like
reinterpreation of subresultants, which provides in particular an 
efficient algorithm for computing them. Since it will play an important 
role in this paper, we take a few lines to recall it.

This reinterpretation is based on the so-called \emph{subresultant 
pseudo-remainder sequence} which is defined as follows. We pick $A$ and 
$B$ as above and assume that $d_A = \deg A$. Denoting by $(P \,\%\, Q)$ 
the remainder in the Euclidean division of $P$ by $Q$, we define two 
recursive sequences $(S_i)$ and $(c_i)$ as follows:
$$\left\{\begin{array}{ll}
S_{-1} = A, \, S_0 = B, \, c_{-1} = 1 \smallskip \\
\displaystyle S_{i+1} = (-s_i)^{\delta_i + 1}
s_{i-1}^{-1} \: c_i^{-\delta_i} \cdot (S_{i-1} \,\%\, S_i)
& \text{for } i \geq 1 \smallskip \\
\displaystyle c_{i+1} = s_{i+1}^{\delta_{i+1}} \cdot c_i^{1-\delta_{i+1}}
& \text{for } i \geq 0. 
\end{array}\right.$$
Here $n_i = \deg S_i$ and $s_i$ is the leading coefficient of 
$S_i$ if $i \geq 0$ and $s_{-1} = 1$ by convention. These sequences 
are finite and the above recurrence applies until $S_i$ has reached 
the value $0$.

\begin{prop}
\label{prop:prem}
With the above notations, we have:
$$\begin{array}{rcll}
\Res_j(A,B) & = & S_i & \text{if } j = n_{i-1} - 1 \smallskip \\
& = & 0 & \text{if } n_i < j < n_{i-1} - 1 < j \\
& = & \!\big(\frac{s_i}{s_{i-1}}\big)^{\delta_i-1} \cdot S_i & 
\text{if } j = n_i
\end{array}$$
for all $i$ such that $S_i$ is defined.
\end{prop}

\begin{rem}
The Proposition \ref{prop:prem} computes \emph{all} subresultants. We
note moreover that, in the common case where $n_{i-1} = n_i - 1$, the
two formulas giving $\Res_{n_i}(A,B)$ agree.

Mimicing ideas behind extended Euclide's algorithm, one can define the 
``extended subresultant pseudo-remainder sequence'' as well and compute
this way at the same time the cofactors.
\end{rem}

Important simplifications occur in the common case where all scalar 
subresultants do not vanish. Under this additional assumption, one can 
prove that the degrees of the $S_i$'s decrease by one at each step, 
meaning that we $\deg S_i = d_B - i$ for all $i$. The sequence 
$(S_i)$ then stops at $i = d_B$. Moreover, the $\delta_i$'s and the 
$c_i$'s are now all ``trivial'' since we have $\delta_i = 1$ and $c_i
= s_i$ for all $i$. The recurrence formula then becomes:
$$S_{i+1} = s_i^2 \cdot s_{i-1}^{-2}
\cdot (S_{i-1} \,\%\, S_i) \quad \text{for } i \geq 1.$$
To conclude with, we remark that Proposition \ref{prop:prem} simply 
states that $R_j = S_{d_B-j}$ in this case.

\section{Euclide-like algorithms\\are unstable}
\label{sec:unstable}

%The aim of this section is to study the loss of precision 

\subsection{A lower bound on losses of precision}

We consider two fixed polynomials $A$ and $B$ with coefficients in $\A$ 
whose coefficients are known with precision $O(p^N)$ for some positive
integer $N$. For simplicity, we assume further that $A$ and $B$ are both
monic and share the same degree $d$. 
For any integer $j$ between $0$ and $d-1$, we denote by $R_j$ the $j$-th 
subresultant of $A$ and $B$.

In this subsection, we estimate the loss of precision if we compute the 
$R_j$'s using the recurrence formula given by Proposition \ref{prop:prem}. 
In what follows, we are going use a \emph{flat precision model}: this 
means that a polynomial $P(X)$ is represented as:
$$P(X) = \sum_{i=1}^n a_i X^i + O(\pi^N)
\quad \text{with } a_i \in K \text{ and } N \in \Z.$$
In other words, we assume that the software we are using does not
carry a precision data on each coefficient but only a unique precision
data for the whole polynomial. Concretely this means that, after having
computing a polynomial, the software truncates the precision on each
coefficients to the smallest precision. 
One can argue that this assumption is too strong (compared to usual
implementations of $p$-adic numbers). Nevertheless, it defines a 
simplified framework where computations can be performed and experiments 
show that it rather well reflects the behaviour of the loss of precision 
in Euclide-like algorithms.

Let $V_j$ be the valuation of the scalar $j$-th subresultant of $(A,B$) 
and $W_j$ be the smallest valuation of a coefficient of $R_j$. We of 
course have $V_j \geq W_j$ and we set $\delta_j = V_j - W_j$.

\begin{prop}
\label{prop:precEuclide}
Let $A$ and $B$ as above. We assume that all scalar subresultants of 
$(A,B)$ do not vanish. Then, in the flat precision model, the 
subresultant pseudo-remainder sequence computes $R_j$ at precision 
$O(\pi^{N_j})$ with:
$$N_j \leq N + V_{j+1} - 2 \cdot (\delta_{j+1} + \delta_{j+2} + \cdots
+ \delta_{d-1}).$$
\end{prop}

\begin{proof}
By our assumptions, the subresultant pseudo-remainder sequence 
is defined by the recurrence:
$$R_{j-1} = r_j^2 \cdot r_{j+1}^{-2} \cdot 
(R_{j+1} \,\%\, R_j).$$
Using that $R_j$ and $R_{j-1}$ have the expected degrees, the
remainder $(R_{j+1} \,\%\, R_j)$ is computed as follows:
$$\begin{array}{rrcl}
\text{we set:} & S & = & R_{j+1} - r_{j+1} \cdot r_j^{-1} \cdot R_j \smallskip \\
\text{and we have:} & R_{j+1} \,\%\, R_j & = & S - s \cdot r_j^{-1} \cdot R_j
\end{array}$$
where $s$ is the coefficient of degree $j$ of $S$. Let us first estimate 
the precision of $S$. Recall for that the relative precision is defined 
as the difference between the (absolute) precision and the valuation: 
intuitively, it is the number of significant digits. It is easy to check 
that the relative precision of a product is the minimum of those of the 
factors. 
Thus the relation precision of $r_{j+1} \cdot r_j^{-1} \cdot R_j$ is
$\min(N_{j+1} - V_{j+1}, N_j - V_j)$. Its absolute precision is then
$M = \min(N_{j+1} - \delta_j, N_j - \delta_j + V_{j+1} - V_j)$. This
quantity is also the precision of $S$ since the other summand
$R_{j+1}$ is known with higher precision. Repeating the argument, we
find that the precision of $(R_{j+1} \,\%\, R_j)$ is equal to 
$\min(M - \delta_j, N_j - \delta_j + \val(s) - V_j)$ and therefore is 
lower bounded by $M - \delta_j \leq N_j - 2 \delta_j + V_{j+1} - V_j$.
From this, we derive $N_{j-1} \leq N_j - 2 \delta_j - V_{j+1} + V_j$ and
the proposition finally follows by summing up these inequalities.
\end{proof}

The difference 
$N - N_0 = - V_1 + 2 \sum_{k=1}^d \delta_j$
is a lower bound on the number of digits lost after having computed the 
resultant using the subresultant pseudo-remainder sequence algorithm. In 
the next subsection (\emph{cf} Corollary \ref{cor:Vj}), we shall see that 
$V_1$ and all $\delta_j$'s are 
approximatively equal to $\frac 1 {p-1}$ on average. The loss of 
precision then grows linearly with respect to $d$. This confirms the 
precision benchmarks shown in Figure \ref{fig:precision}.
We emphasize one more time that this loss of precision is \emph{not}
intrinsic but an artefact of the algorithm we have used; indeed, one
should not loose any precision when computing resultants because they
are given by polynomial expressions.

\subsection{Behaviour on random inputs}

Proposition \ref{prop:precEuclide} gives an estimation of the loss
of precision in Euclide-like algorithms in terms of the quantities
$V_j$ and $\delta_j$. It is nevertheless \emph{a priori} not clear
how large these numbers are. The aim of this paragraph is to compute
their order of magnitude when $A$ and $B$ are picked randomly among
the set of monic polynomials of degree $d$ with coefficients in $\A$.
In what follows, we assume that the residue field of $\A$ is finite
and we use the letter $q$ to denote its cardinality.

We endow $\A$ with its Haar measure. The set $\Omega$ of couples of 
monic polynomial of degree $d$ with coefficients in $\A$ is canonically 
in bijection with $\A^{2d}$ and hence inherits the product measure. 
We consider $V_j$, $W_j$ and $\delta_j$ as random variables defined on 
$\Omega$.

\begin{theo}
\label{th:lawVj}
We fix $j \in \{0, \ldots, d-1\}$.
Let $X_0, \ldots, X_{d-1}$ be $d$ pairwise independant discrete random 
variables with geometric law of parameter $(1 - q^{-1})$, \emph{i.e.}
$$\P[X_i = k] = (1 - q^{-1}) \cdot q^{-k} \quad 
\text{(with } 0 \leq i < d \text{ and } k \in \N \text{)}.$$
Then $V_j$ is distributed as the random variable
\begin{center}
$\displaystyle Y_j = \sum_{i=0}^d \min(X_{j-i}, X_{j-i+1}, \ldots, X_{j+i})$
\end{center}
\noindent
with $X_i = +\infty$ if $i < 0$ and $X_i = 0$ if $i \geq d$.
\end{theo}

\begin{rem}
The above Theorem does not say anything about the correlations between
the $X_j$'s. In particular, we emphasize that it is \emph{false} that
the tuple $(V_{d-1}, \ldots, V_0)$ is distributed as $(Y_{d-1}, \ldots,
Y_0)$. For instance, one can prove that $(V_{d-1}, 
V_{d-2})$ is distributed as 
$(X, \, X' + \min(X', [X/2]))$
where $X$ and $X'$ are two independant discrete random variables with 
geometric law of parameter $(1 - q^{-1})$ and the notation $[\cdot]$ 
stands for the integer part function. In particular, we observe that 
$(V_{d-2}, V_{d-1}) \neq (2,1)$ almost surely although the events 
$\{V_{d-1} = 2\}$ and $\{V_{d-2} = 1\}$ both occur with positive 
probability.

Nonetheless, a consequence of Proposition \ref{prop:distribk} below is 
that the variables $\bar V_j = \mathbbm{1}_{\{V_j = 0\}}$ are mutually 
independant.
\end{rem}

\begin{theo}
\label{th:deltaj}
For all $j \in \{0, \ldots, d-1\}$ and all $m \in \N$, we have:
$$\P[\delta_j \geq m] \geq \frac{(q-1)(q^j-1)}{q^{j+1}-1} q^{-m}.$$
\end{theo}

The proof of these two Theorem is given in \S \ref{subsec:proof}. 
We now derive some consequences. Let $\sigma$ denote the following
permutation:
$$\begin{array}{cl}
\left(
\begin{array}{cccccccc}
1 & 2 & \cdots & \frac d 2 & \frac d 2 + 1 & \frac d 2 + 2 & \cdots & d \smallskip \\
1 & 3 & \cdots & d-1 & d & d-2 & \cdots & 2 
\end{array} \right) & \text{if } 2 \mid d \medskip \\
\left(
\begin{array}{cccccccc}
1 & 2 & \cdots & \frac{d+1} 2 & \frac{d+3} 2 & \frac{d+5} 2 & \cdots & d \smallskip \\
1 & 3 & \cdots & d & d-1 & d-3 & \cdots & 2 
\end{array} \right) & \text{if } 2 \nmid d.
\end{array}$$
In other words, $\sigma$ takes first the odd values in $[1,d]$ in 
increasing order and then the even values in the same range in 
decreasing order.

\begin{cor}
\label{cor:Vj}
For all $j \in \{0, \ldots, d-1\}$, we have:

\begin{enumerate}[\hspace{0.3cm}(1)]
\setlength\itemsep{0.1em}
\item $\E[V_j] = 
\displaystyle \sum_{i=1}^{d-j} \frac 1 {q^{\sigma(i)} - 1}$ \\
in particular $\frac 1 {q-1} \leq \E[V_j] < \frac q {(q-1)^2}$
\item $\frac {q^j - 1}{q^{j+1} - 1} \leq \E[\delta_j] \leq \E[V_j]$
\item $\sigma[V_j]^2 = 
\displaystyle \sum_{i=1}^{d-j} \frac {(2i-1) \cdot q^{\sigma(i)}}
{(q^{\sigma(i)} - 1)^2}$ \\
in particular $\frac{\sqrt q}{q-1} \leq \sigma[V_j] < 
\frac {q \:\sqrt{q+1}} {(q-1)^2}$
\item $\P[V_j \geq m] \leq q^{-m + O(\sqrt m)}$
\item $\E[\max(V_0, \ldots, V_{d-1})] \leq \log_q d + O(\sqrt{\log_q d})$
\end{enumerate}
\end{cor}

\begin{proof}
By Theorem \ref{th:lawVj}, the expected value of $V_j$ is equal to 
$\sum_{i=0}^d \E[Z_i]$ with $Z_i = \min(X_{j-i}, \ldots, X_{j+i})$ ($j$ 
is fixed during all the proof). Our conventions imply that $Z_i$ 
vanishes if $i > d-j$. On the contrary, if $i \leq d-j$, let us define 
$\tau(1), \ldots, \tau(d-j)$ as the numbers $\sigma(1), \ldots, 
\sigma(d-j)$ sorted in increasing order. The random variable $Z_i$ 
is then the minimum of $\tau(i)$ independant random variables with 
geometric distribution of parameter $(1 - q^{-1})$ and thus its 
distribution is geometric of parameter $(1 - q^{-\tau(i)})$.
Its expected value is then $\frac 1 {q^{\tau(i) - 1}}$ and the first
formula follows. The inequality $\frac 1 {q-1} \leq \E[V_j]$ is clear
because $\frac 1 {q-1}$ is the first summand in the expansion of
$\E[V_j]$. The upper bound is obtained as follows:
$$\E[V_j] < \sum_{i=0}^\infty \frac 1{q^i - 1}
\leq \sum_{i=0}^\infty \frac 1{q^i - q^{i-1}} = \frac q {(q-1)^2}.$$

The first inequality of claim~(2) is obtained from the relation 
$\E[\delta_j] = \sum_{m=1}^\infty \P[\delta_j \geq m]$ using the 
estimation of Theorem \ref{th:deltaj}. The second inequality is 
clear because $\delta_j \leq V_j$.

The variance of $V_j$ is related to the covariance of $Z_i$'s thanks to 
the formula
$\Var(V_j) = \sum_{1 \leq i,i' \leq d-j} \Cov(Z_i, Z_{i'})$.
Moreover, given $X$ and $X'$ two independant variables having geometric
distribution of parameter $(1 - a^{-1})$ and $(1 - b^{-1})$ respectively,
a direct computation gives:
$$\Cov(X, \min(X,X')) = \frac{ab}{(ab-1)^2}.$$
Applying this to our setting, we get:
$$\Cov(Z_i, Z_{i'}) = \frac{q^{e(i,i')}}{(q^{e(i,i')}-1)^2}$$
where $e(i,i') = \min(\tau(i), \tau(i')) = \tau(\min(i,i'))$. Summing
up these contributions, we get the equality in~(3). The inequalities
are derived from this similarly to what we have done in~(1).

Let $(Z_i)_{i \geq 0}$ be a countable family 
of independant random variable having all geometric distribution of 
parameter $(1 - q^{-1})$. We set $Z = \sum_{i=1}^\infty \min(Z_1, \ldots, 
Z_i)$. Cleary $V_j \leq Z$ and it is then enough to prove:
$$\P[Z \geq m] \leq q^{-m + O(\sqrt m)}.$$
We introduce the event $E_m$ formulated as follows:
\emph{there exists a partition $(m_1, \ldots, m_\ell)$ of $m$ such that $X_i 
\geq m_i$ for all $i \leq \ell$}.
Up to a neglectable subset, $E_m$ contains the event $\{ Z \geq m\}$.
Thus $\P[Z \geq m] \leq \P[E_m]$. Furthermore, the probability of $E_m$
is bounded from above by $\sum \prod_{i=1}^\ell \P[X_1 \geq m_i]$ where
the sum runs over all partitions $(m_1, \ldots, m_\ell)$ of $m$. 
Replacing $\P[X_1 \geq m_i]$ by $q^{-m_i}$, we get
$\P[E_m] \leq p(m) \cdot q^{-m}$
where $p(m)$ denotes the number of partitions of $m$. By a famous 
formula \cite{andrews}, we know that $\log p(m)$ is equivalent to $\pi 
\sqrt{2m/3}$. In particular it is in $q^{O(\sqrt m)}$ and~(4) is proved.

We now derive~(5) by a standard argument. It follows from~(4) that
$\P[\max(V_0, \ldots, V_{d-1})] \leq d \cdot q^{-m + c\sqrt m}$
for some constant $c$. Therefore:
$$\E[\max(V_0, \ldots, V_{d-1})] \leq \sum_{m=1}^\infty \min(1,
d \cdot q^{-m + c\sqrt m}).$$
Let $m_0$ denote the smallest index such that $d \cdot q^{-m_0 + c\sqrt 
m_0}$, \emph{i.e.} $m_0 - c \sqrt{m_0} \geq \log_q d$. Solving the latest
equation, we get $m_0 = \log_q + O(\sqrt{\log_q d})$. Moreover
$\sum_{m=m_0}^\infty d\: q^{-m + c\sqrt m}$ is bounded independantly of
$d$. The result follows.
\end{proof}

\subsection{Proof of Theorems \ref{th:lawVj} and \ref{th:deltaj}}
\label{subsec:proof}

During the proof, $A$ and $B$ will always refer to monic polynomials of 
degree $d$ and $R_j$ (resp. $U_j$ and $V_j$) to their $j$-th 
subresultant (resp. their $j$-th cofactors). If $P$ is a polynomial and 
$n$ is a positive integer, we use the notation $P[n]$ to refer to the 
coefficient of $X^n$ in $P$. We set $r_j = R_j[j]$.

\medskip

\noindent
\textbf{Preliminaries on subresultants.}
We collect here various useful relations between subresultants and 
cofactors. During all these preliminaries, we work over an arbitrary
base $\ring$.

\begin{prop}
\label{prop:relations}
The following relations hold:
\begin{itemize}
\setlength\itemsep{0.1em}
\item $U_{j-1} V_j - U_j V_{j-1} = (-1)^j r_j^2$;
\item $U_j[d{-}j{-}1] = -V_j[d{-}j{-}1] = (-1)^j r_{j+1}$;
\item $\Res^{j,j-1}_k(R_j, R_{j-1}) = r_j^{2(j-k-1)} \: R_k$ for $k < j$;
\item $\Res^{d-j,d-j-1}_k(U_{j-1}, U_j) = r_j^{2(d-j-k-1)} \: U_{d-1-k}$\\
for $k < d-j$.
\end{itemize}
Moreover $r_j$ depends only on the $2j$ coefficients of highest 
degree of $A$ and $B$.
\end{prop}

\begin{proof}
By functoriality of subresultants, we may assume that $\ring = 
\Z[a_0, \ldots, a_{d-1}, b_0, \ldots, b_{d-1}]$ and that $A$ and $B$
are the two generic monic polynomials 
$A = X^d + \sum_{i=0} a_i X^i$ and
$B = X^d + \sum_{i=0} b_i X^i$.
All relations then follows from the extended subresultants 
pseudo-remainder sequence algorithm (noting in addition that all
scalar subresultants are nonzero).
\end{proof}

For any fixed index $j \in \{1, \ldots, d-1\}$, we consider the 
function $\psi_j$ that takes a couple $(A,B) \in \ring_d[X]^2$ to the
quadruple $(U_j, U_{j-1}, R_j, R_{j-1})$. It follows from Proposition
\ref{prop:relations} that $\psi_j$ takes its values in the subset 
$\mathcal E_j$ of
$$\big(\ring_{\leq d-j-1}[X]\big) \times \big(\ring_{\leq d-j}[X]\big) \times 
\big(\ring_{\leq j}[X]\big) \times \big(\ring_{\leq j-1}[X]\big)$$
gathering the quadruples $(\mathcal U_j, \mathcal U_{j-1}, \mathcal
R_j, \mathcal R_{j-1})$ such that:
$$\begin{array}{ll}
&\mathcal U_{j-1}[d{-}j] = (-1)^{j-1} \: \mathcal R_j[j] \smallskip \\
\text{and} &
\Res^{d-j,d-j-1}(\mathcal U_{j-1}, \mathcal U_j) = -\mathcal R_j[j]^{2(d-j-1)}.
\end{array}$$
Let $\mathcal E_j^\times$ be the subset of $\mathcal E_j$ defined by 
requiring that $\mathcal R_j[j]$ is invertible in $\ring$. In the same 
way, we define $\Omega_j^\times$ as the subset of $\ring_d[X]^2$ 
consisting of couples $(A,B)$ whose $j$-th scalar subresultants (in 
degree $(d,d)$) is invertible in $\ring$.

\begin{prop}
\label{prop:bijection}
With the above notations, the function $\psi_j$ induces a bijection between
$\Omega_j^\times$ and $\mathcal E_j^\times$.
\end{prop}

\begin{proof} 
We are going to define the inverse of $\psi_j$. We fix a quadruple 
$(\mathcal U_j, \mathcal U_{j-1}, \mathcal R_j, \mathcal R_{j-1})$ in 
$\mathcal E_j^\times$ and set $a = \mathcal R_j[j]$.
Let $\mathcal W_j$ and $\mathcal W_{j-1}$ denote 
the $j$-th cofactors of $(\mathcal U_{j-1}, \mathcal U_j)$ in degree 
$(d{-}j, d{-}j{-}1)$. Define $\mathcal V_j = \alpha \mathcal W_j$ and 
$\mathcal V_{j-1} = -\alpha \mathcal W_{j-1}$ where $\alpha =
a^{4j-4d+6}$. The relation:
\begin{equation}
\label{eq:missingcof}
\mathcal U_{j-1} \mathcal V_j - 
\mathcal U_j \mathcal V_{j-1} = a^2.
\end{equation}
then holds. We now define $A$ and $B$ using the formulae:
\begin{equation}
\label{eq:systemAB}
\left\{\begin{array}{l}
A = (-1)^j \cdot a^{-2} \cdot (V_j R_{j-1} - V_{j-1} R_j) \smallskip \\
B = (-1)^{j-1} \cdot a^{-2} \cdot (U_j R_{j-1} - U_{j-1} R_j) \\
\end{array}\right.
\end{equation}
and let $\varphi_j$ be the function mapping $(\mathcal U_j, \mathcal 
U_{j-1}, \mathcal R_j, \mathcal R_{j-1})$ to $(A,B)$. The composite 
$\varphi_j \circ \psi_j$ is easily checked to be the identity: indeed, 
if $\psi_j(A,B) = (\mathcal U_j, \mathcal U_{j-1}, \mathcal R_j, 
\mathcal R_{j-1})$, the relation~\eqref{eq:missingcof} implies that 
$\mathcal V_{j-1}$ and $\mathcal V_j$ are the missing cofactors and, 
consequently, $A$ and $B$ have to be given by the 
system~\eqref{eq:systemAB}.

To conclude the proof, it remains to prove that the composite in 
the other direction $\psi_j \circ \varphi_j$ is the identity as well. 
Since both $\varphi_j$ and $\psi_j$ are componant-wise given by 
polynomials, we can use functoriality and assume that $\ring$ is 
the field $\Q(c_0, c_1, \ldots, c_n)$ (with $n = 2d$) and that 
each variable $c_i$ corresponds to one coefficient of $\mathcal U_j$, 
$\mathcal U_{j-1}$, $\mathcal R_j$ and $\mathcal R_{j-1}$ with the
convention that $c_0$ (resp. $(-1)^{j-1} c_0$) is used for the leading
coefficients of $\mathcal R_j$ (resp. $\mathcal U_{j-1}$). Set:
$$\begin{array}{ll}
& (A,B) = \varphi_j(\mathcal U_j, \mathcal U_{j-1}, 
\mathcal R_j, \mathcal R_{j-1}) \smallskip \\
\text{and} & (U_j, U_{j-1}, R_j, R_{j-1}) = \psi_j(A,B)
\end{array}$$
Since $\ring$ is a field and $\mathcal R_j[j]$ does not vanish, the 
Sylvester mapping
$$\begin{array}{rcl}
\ring_{<d-j}[X] \times \ring_{<d-j}[X] & \to & 
\ring_{<2d-j}[X]/\ring_{<j}[X] \smallskip \\
(U,V) & \mapsto & AU + BV
\end{array}$$
has to be bijective. Therefore there must exist $\lambda \in \ring$ 
such that $\mathcal R_j = \lambda \cdot R_j$ and $\mathcal U_j = \lambda 
\cdot U_j$. Similarly $(\mathcal R_{j-1}, \mathcal U_{j-1}) = \mu \cdot
(R_{j-1}, U_{j-1})$ for some $\mu \in \ring$. Identifying the leadings 
coefficients, we get $\lambda = \mu$. Writing
$\Res^{d-j,d-j-1}(\mathcal U_{j-1}, \mathcal U_j) = 
\Res^{d-j,d-j-1}(U_{j-1}, U_j)$,
we get $\lambda^{2(d-j)-1} = 1$. Since the exponent is odd, this implies 
$\lambda = 1$ and we are done.
\end{proof}

\begin{cor}
\label{cor:bijection}
We assume that $\ring = \A$. Then the map $\psi_j : \Omega_j^\times
\to \mathcal E_j^\times$ preserves the Haar measure.
\end{cor}

\begin{proof}
Proposition \ref{prop:bijection} applied with the quotient rings $\ring 
= \A/\pi^n\A$ shows that $(\psi_j \text{ mod } \pi^n)$ is a bijection 
for all $n$. This proves the Corollary.
\end{proof}

\noindent 
\textbf{The distribution in the residue field.}
We assume in this paragraph that $\ring$ is a finite field of 
cardinality $q$. We equip $\Omega_\ring = \ring_d[X]^2$ with the uniform 
distribution. For $j \in \{0, \ldots, d-1\}$ and $(A,B) \in \Omega_\ring$,
we set $\bar V_j(A,B) = 1$ if $r_j(A,B)$ vanishes and $\bar V_j(A,B) = 0$
otherwise. The functions $\bar V_j$'s define random variables over 
$\Omega_\ring$.

\begin{prop}
\label{prop:distribk}
With the above notations, the $\bar V_j$'s are mutually independant
and they all follow a Bernoulli distribution of parameter $\frac 1 q$.
\end{prop}

\begin{proof} 
Given a subset $J$ of $\{0, \ldots, d-1\}$, we denote by 
$\Omega_\ring(J)$ the subset of $\Omega_\ring$ consisting of those 
couples $(A,B)$ for which $r_j(A,B)$ does not vanish if and only if $j 
\in J$. We want to prove that $\Omega_\ring(J)$ has cardinality 
$q^{2d-\Card J} (q-1)^{\Card J}$. To do this, we introduce several 
additional notations. First, we fix $J$ and write $J = \{n_1, \ldots, 
n_\ell\}$ with $n_1 > n_2 > \cdots > n_\ell$. We also set $n_{\ell+1} = 
0$ by convention. Then, given $n$ and $m$ two integers with $m < n$, let 
$V_{(m,n)}$ denote the set of polynomials of the form
$a_m X^m + a_{m+1} X^{m+1} \cdots + a_n X^n$
with $a_i \in \ring$ and $a_n \neq 0$. Clearly, $V_{n,m}$ has
cardinality $(q-1) q^{n-m}$. If $P$ is any polynomial of 
degree $n$ and $m < n$ is an integer, we further define $P[m{:}] \in 
V_{(m,n)}$ as the polynomial obtained from $P$ by removing its monomials 
with coefficients $< m$. Third, given $(A,B)$ in $\Omega_\ring$, we 
denote by $(S_i(A,B))$ its subresultant pseudo-remainder sequence as 
defined in \S \ref{subsec:subres}. We note that, if $(A,B) \in 
\Omega_\ring(J)$, the sequence $(S_i(A,B))$ stops at $i = \ell$ and we 
have $\deg S_i = n_i$ for all $i$. We now claim that the mapping
$$\begin{array}{rcl} 
\Lambda_J : \,\, 
\Omega_\ring(J) & \to & 
V_{(n_1,n_2)} \times \cdots \times V_{(n_\ell,n_{\ell+1})} \smallskip \\
(A,B) & \mapsto & 
\big(S_i(A,B)[n_{i+1}{:}]\big)_{1 \leq i \leq \ell}
\end{array}$$ 
is injective. In order to establish the claim, we remark that the 
knowledge of $S_{i-1}(A,B)$ and $S_i(A,B)[n_{i+1}{:}]$ (for some $i$) 
is enough to reconstruct the quotient of the Euclidean division of 
$S_i(A,B)$ by $S_{i-1}(A,B)$. Thus, one can recover $S_i(A,B)$ from 
$S_{i-2}(A,B)$, $S_{i-1}(A,B)$ and $S_i(A,B)[n_{i+1}{:}]$. We deduce 
from this that $\Lambda_J(A,B)$ determines uniquely all $S_i(A,B)$'s 
and finally $A$ and $B$ themselves. This proves the claim.

Finally, we note that the claim implies that the cardinality of 
$\Omega_\ring(J)$ is at most $q^{2d-\ell} (q-1)^\ell$. Summing up these 
inequalities over all possible $J$, we get $\Card \Omega_\ring \leq 
q^{2d}$. This latest inequality being an equality, we need to have
$\Card \Omega_\ring(J) = q^{2d-\Card J} (q-1)^{\Card J}$ for all $J$.
\end{proof}

\noindent
\textbf{Proof of Theorem \ref{th:deltaj}.}
We assume first that $j < d-1$. Proposition \ref{prop:distribk} above 
then says that $r_{j+1}$ is invertible in $\A$ with probability 
$(1 - q^{-1})$. Moreover, assuming that this event holds, Corollary 
\ref{cor:bijection} implies that $R_j$ is distributed in $\A_{\leq 
j}[X]$ according to the Haar measure. An easy computation gives
$\P[\delta_j \geq m\,|\, r_{j+1} \in \A^\times ] =
\frac{q(q^j - 1)}{q^{j+1}-1}$
and therefore:
$$\P[\delta_j \geq m] \geq (1 - q^{-1}) \cdot
\frac{q(q^j - 1)}{q^{j+1}-1} =
\frac{(q-1)(q^j - 1)}{q^{j+1}-1}.$$
The case $j = d-1$ is actually simpler. Indeed, the same argument works 
expect that we know for sure that $r_{j+1} = r_d$ is invertible since it 
is equal to $1$ by convention. In that case, the probability is then 
equal to $\frac{q(q^j - 1)}{q^{j+1}-1}$.

\medskip

\noindent
\textbf{Proof of Theorem \ref{th:lawVj}.}
We fix $j \in \{0, \ldots, d-1\}$. 
We define a random variable $V_j^{(0)}$ as the greatest (nonnegative) 
integer $v$ such that all scalar subresultants $r_{j'}$ have positive 
valuation for $j'$ varying in the open range $(j-v, j+v)$ (with the
convention that $r_{j'} = 0$ whenever $j' < 0$). It is clear from the 
definition that $r_{j-v}$ or $r_{j+v}$ (with $v = V_j^{(0)}$) has 
valuation $0$. Moreover, assuming first that $\val(r_{j+v}) = 0$, we 
get by Proposition \ref{prop:relations}:
$$\begin{array}{rl}
& \val(r_j) = v + \val\big(r_v^{j-v,j-v+1}(A^{(0)}, B^{(0)})\big) 
\smallskip \\
\text{with} &
A^{(1)} = \frac 1 {r_{j+v} X^{j-v-1}} \cdot R_{j+v}[j{-}v{-}1\:{:}], \smallskip \\
\text{and} &
B^{(1)} = A^{(1)} + \frac 1 {\pi X^{j-v-1}} \cdot R_{j+v-1}[j{-}v{-}1\:{:}].
\end{array}$$
We notice that all the coefficients of $B^{(1)}$ lie in $\A$ 
because $r_{j'}$ has positive valuation for $j' \in (j-v, j+v)$.
Furthermore, Corollary \ref{cor:bijection} shows that the 
couple $(A^{(1)}, B^{(1)})$ is distributed according to the Haar measure 
on $(\A_{2v-1}[X])^2$. If $\val(r_{j+v}) = 0$, one can argue similarly 
by replacing $R_{j+v}$ and $R_{j+v-1}$ by the cofactors $U_{j-v}$ and 
$U_{j-v+1}$ respectively. Replacing $(A,B)$ by $(A^{(1)}, B^{(1)})$, we 
can now define a new random variable $V_j^{(1)}$ and, continuing this 
way, we construct an infinite sequence $V_j^{(m)}$ such that
$V_j = \sum_{m \geq 0} V_j^{(m)}$.

We now introduce a double sequence $(X_i^{(m)})_{0 \leq i < d, m 
\geq 0}$ of mutually independant random variables with Bernoulli
distribution of parameter $\frac 1 q$ and we agree to set 
$X_{j'}^{(m)} = 0$ for $j' < 0$ and $X_{j'}^{(m)} = 1$ for $j \geq d$.
It follows from Proposition \ref{prop:distribk} (applied with $\ring = 
k$) that $V_j^{(0)}$ has the same distribution than 
$Y_j^{(0)} = \sum_{i=1}^d \min(X_{j-i}^{(0)}, \ldots, X_{j+i}^{(0)})$.
In the same way, keeping in mind that $A^{(1)}$ and $B^{(1)}$ have both 
degree $2V_j^{(0)} - 1$, we find that $V_j^{(1)}$ has the same 
distribution than $\sum_{i=1}^{V_j^{(0)} - 1} \min(X_{j-i}^{(1)}, 
\ldots, X_{j+i}^{(1)})$, which can be rewritten as
$Y_j^{(1)} = \sum_{i=1}^d \min(X_{j-i}^{(0)}, X_{j-i}^{(1)}, \ldots, 
X_{j+i}^{(0)}, X_{j+i}^{(1)})$. More precisely,
the equidistribution of $(A^{(1)}, B^{(1)})$ shows that the 
joint distribution $(V_j^{(0)}, V_j^{(1)})$ is the same as those
of $(Y_j^{(0)}, Y_j^{(1)})$. Repeating the argument, we see that
$(V_j^{(m)})_{m \geq 0}$ is distributed as $(Y_j^{(m)})_{m \geq 0}$
where:
$$Y_j^{(m)} = \sum_{i=1}^d \min(X_{j-i}^{(0)}, \ldots X_{j-i}^{(m)}, 
\ldots, X_{j+i}^{(0)}, \ldots, X_{j+i}^{(m)}).$$
Setting finally $X_i = \sum_{m \geq 0} \min(X_1^{(0)}, \ldots, 
X_i^{(m)})$, we find the $X_i$'s ($0 \leq i < d$) are mutually 
independant and that they all follow a geometric distribution of 
parameter $(1 - q^{-1})$. We now conclude the proof by noting that 
$Y_j$ equals $\sum_{i=1}^d \min(X_{j-i}, \ldots, X_{j+i})$ (recall
that the $X_i^{(m)}$'s only take the values $0$ and $1$).

\section{The method of adaptative lifts}
\label{sec:stable}

We have seen in the previous sections that Euclide's algorithm (and 
their variants) are very unstable in practice. On the other hand, 
one can compute subresultants in a very stable way by evaluating
the corresponding minors of the Sylvester matrix. Doing so, we do
not loose any significant digit. Of course, the downside is the
quite bad efficiency.

In this section, we design an algorithm which combines the two
advantages: it has the same complexity than Euclide's algorithm
and it is very stable in the sense that it does not loose any
significant digit. 
This algorithm is deduced from the subresultant pseudo-remainder
sequence algorithm by applying a ``stabilization process'', whose 
inspiration comes from \cite{padicprec}.

\subsection{Crash course on ultrametric precision}

In this subsection, we briefly report on the results of \cite{padicprec}. In this 
paper, the authors draw the lines of a general framework to handle a 
sharp (often optimal) track of ultrametric precision. In order to do so, 
they proposed to attach a single precision datum --- which is a lattice 
in a huge vector space --- to the set of all variables, instead of 
attaching a precision datum to each individual variable as we do 
usually. This proposal is supported by the following lemma in ultrametric
analysis.

\begin{lem}
\label{lem:padicprec}
Let $f : K^n \to K^m$ be a strictly differentiable function. Let $x$
be a point in $K^n$ at which the differential of $f$ is surjective.
Then, for all sufficiently large $N$, the following equality holds:
\begin{equation}
\label{eq:padicprec}
f(x + p^N \A^n) = f(x) + f'(x)(p^N \A^n)
\end{equation}
where $f'(x)$ stands for the differential of $f$ at $x$.
\end{lem}

An important feature of Lemma \ref{lem:padicprec} is that 
Eq.~\eqref{eq:padicprec} is an equality, and not only an inclusion. 
This reflects the optimality of the method and, as we shall see later, 
is also the origin of the ``stabilization process'' mentionned above.
Moreover, if the function $f$ appearing in Lemma \ref{lem:padicprec} 
has more regularity, one can exhibit a lower bound on $N$ ensuring
that Eq.~\eqref{eq:padicprec} holds. In \cite{padicprec}, the case of locally 
analytic functions is considered. For our purpose, it will be enough 
to restrict ourselves to \emph{integral polynomial} functions, that
are functions of the shape:
$$f(x) = \big(F_1(x), \ldots, F_m(x)\big)
\quad \text{with } x = (x_1, \ldots, x_n)$$
where $F_1, \ldots, F_m$ are multivariate polynomials in $n$ variables
with coefficients in $\A$. Under this assumption, we have:

\begin{prop}
\label{prop:padicprec}
Let $f : K^n \to K^m$ be an integral polynomial function and $x \in
\A^n$. Let $v$ be an integer such that $f'(x)(\A^n) \supset p^v \A^m$.
Then, for all $N > v$, Eq.~\eqref{eq:padicprec} holds.
\end{prop}

\begin{proof}
It is a direct corollary of \cite[Proposition 3.12]{}.
\end{proof}

\subsection{The stabilization process}

In order to get a stabilized version of the subresultant 
pseudo-remainder sequence algorithm, we are going to estimate 
the loss --- and the gain --- of precision at each step of this 
algorithm using Proposition \ref{prop:padicprec}. In order to do
so, we introduce the functions
$$\begin{array}{rcl}
f_j \, : \, K_d[X] \times K_d[X] & \to & K_{\leq j}[X] \times 
K_{\leq j-1}[X] \smallskip \\
(A,B) & \mapsto & (\Res_j(A,B), \Res_{j-1}(A,B)).
\end{array}$$

\noindent
{\bf Differential computation.}
The first step consists in estimating the differential of $f_j$. 
Set $\mathcal L = \A_d[X] \times \A_d[X]$ and $\mathcal L_j = 
\A_{\leq j}[X] \times \A_{\leq j-1}[X]$; they are standard lattices
in the source and the target of $f_j$ respectively.

\begin{lem}
\label{lem:diff}
For all $(A,B) \in K_d[X]^2$, we have:
$$r_j^2 \cdot \mathcal L_j 
\subset f'_j(A,B)(\mathcal L) \subset \mathcal L_j$$
where $r_j$ is the $j$-th scalar subresultant of $(A,B)$.
\end{lem}

\begin{proof}
The second inclusion is clear because $f_j$ is a polynomial function. 
Let us prove the first inclusion. One may of course assume that $r_j$ 
does not vanish, otherwise there is nothing to prove. Now, we remark 
that $f_j$ factors through the function $\psi_j$ introduced in \S 
\ref{subsec:proof}. By continuity, the $j$-th scalar subresultant 
function does not vanish on a neighborhood of $(A,B)$. By Proposition 
\ref{prop:bijection}, $\psi_j$ is injective on this neighborhood. 
Therefore so is $f_j$. Furthermore, a close look at the proof of 
Proposition \ref{prop:bijection} indicates that a left inverse of $f_i$ 
is the function mapping $(S_j, S_{j-1})$ to
$$(-1)^j \cdot r_j^{-2} \cdot 
(V_j R_{j-1}{-}V_{j-1} R_j,\: -U_j R_{j-1}{+}U_{j-1} R_j)$$
where $U_j, V_j$ (resp. $U_{j-1}$, $V_{j-1}$) are the $j$-th (resp 
$(j-1)$-th) cofactors of $(A,B)$. Differenting this, we get the
announced result.
\end{proof}

\noindent
{\bf The algorithm.}
The input of our algorithm is a couple of polynomials $(A,B)$ with 
coefficients in $\A$ whose coefficients are known at precision 
$O(\pi^N)$ for some given positive integer $N$. We make the following 
hypothesis (and refer to \S \ref{subsec:questions} for a brief 
discussion about it):

\medskip

\begin{tabular}{rl}
& (a)~$A$ and $B$ are monic of the same degree $d$ \\
{\bf (H)}: & (b)~all scalar subresultants of $(A,B)$ \\
& \phantom{(b)~}do not vanish modulo $\pi^{N/2}$.
\end{tabular}

\medskip

\noindent
The subresultants $R_j = \Res_j(A,B)$ can then be computed recursively:
setting $R_d = B$, we have $R_{d-1} = A-B$ and the recurrence
$R_{j-2} = r_{j-1}^2 \cdot r_j^{-2} \cdot (R_j \,\%\, R_{j-1})$.
Moreover Lemma \ref{lem:diff} and Proposition \ref{prop:padicprec} imply
that any perturbation of $(R_j, R_{j-1})$ by an element of $p^N r_j^2 
\mathcal L_j$ will not affect the values of the next $R_k$'s modulo 
$\pi^N$. Indeed, given $k > j$, let $f_{j,k} : (R_j, R_{j-1}) \mapsto 
(R_k, R_{k+1})$ be the function modeling the above recurrence. We have
$f_k = f_{j,k} \circ f_j$ and thus we can write:
\begin{align*}
f_{j,k}\big((R_j, R_{j-1}) + \pi^N r_j^2 \mathcal L_j\big) 
& \subset f_k\big((A,B) + \pi^N \mathcal L\big) \\
& \subset (R_k, R_{k+1}) + \pi^N \mathcal L_k.
\end{align*}
This suggests the following algorithm.

\noindent\hrulefill

\noindent {\bf Algorithm 1:} {\tt subresultants\_stable}

\noindent{\bf Input:} an integer $N$, a couple $(A,B)$ satisfying {\bf (H)}

\noindent{\bf Output:} the subresultants of $(A,B)$ at precision $O(\pi^N)$

\smallskip\noindent 1.\ Set $R_d = B$, $v_d = 0$ and $R_{d-1} = A-B$

\smallskip\noindent 2.\ {\bf for} $j=d,\dots,2$

\smallskip\noindent 3.\ \hspace{0.3cm}{\bf compute} $v_{j-1} = \val(\lc(R_{j-1}))$

\smallskip\noindent 4.\ \hspace{0.3cm}{\bf lift} $(R_j, R_{j-1})$ at precision $O(\pi^{N+2(v_j + v_{j-1})})$

\smallskip\noindent 5.\ \hspace{0.3cm}{\bf compute} $R_{j-2} = r_{j-1}^2 \cdot r_j^{-2} \cdot (R_j \,\%\, R_{j-1})$\\
\phantom{6.\ }\hspace{0.3cm}at precision $O(\pi^{N+2 v_{j-1}})$

\smallskip\noindent 6.\ {\bf return} $R_{d-1} + O(\pi^N), \ldots,
R_0 + O(\pi^N)$

\vspace{-1ex}\noindent\hrulefill

\begin{prop}
Algorithm 1 is correct and runs in
$$O(d^2 \cdot \M(N + \max(V_0, \ldots, V_{d-1}))$$
bit operations where $V_j$ denotes the valuation of $r_j$ and 
$\M(n)$ is the number of bit operations needed to perform an 
arithmetic operation (addition, product) in $\A$ at precision 
$O(\pi^n)$.
\end{prop}

\begin{rem}
\label{rem:M}
In all usual examples ($p$-adic numbers, Laurent series), one can 
choose $\M(n)$ is quasi-linear in $n$ and the size of the residue
field $k$.
\end{rem}

\begin{proof}
Regarding to the correctness, it remains only to justify that $R_{j-2}$ 
can be computed at the given precision in line 5 but this follows 
from a naive track of precision (\emph{cf} proof of Proposition
\ref{prop:precEuclide} for a similar computation).
As usual Euclide's algorithm, Algorithm 1 requires $O(d^2)$ operations 
in the base ring $\A$. Moreover, we observe that the maximal precision 
at which we are computing is upper bounded by $N + 2 \max(V_0, \ldots, 
V_{d-1})$. The proposition follows.
\end{proof}

According to Corollary \ref{cor:Vj}, the expected value of the variable 
$\max(V_0, \ldots, V_{d-1})$ is in $O(\log_p d)$. Thus, the average 
complexity of Algorithm 1 is $O(d^2 \cdot \M(N + \log d))$ bit 
operations. In all usual cases (\emph{cf} Remark \ref{rem:M}), this
complexity is also $\tilde O(d^2 N \cdot \log |k|)$ bit operations.

\subsection{Open questions}
\label{subsec:questions}

Of course, the first question we would like to raise is to possibility 
to drop the hypothesis {\bf (H)}. It is actually rather easy to relax
the part~(a) of the hypothesis, replacing it by:

\medskip

(a')~$\lc(B)$ is invertible in $\A$ and $\deg A \geq \deg B$.

\medskip

\noindent
Assuming only this weaker assumption, Algorithm 1 still works except 
that, in line 1, we have to set
$$R_{d-1} = (-1)^{\deg A - \deg B} (A \,\%\, B)$$
(instead of $R_{d-1} = A-B$).
We remark that this value can be computed at precision $O(\pi^N)$
because the leading coefficient of $B$ is assume to be invertible.
Dropping entirely~(a') seems nevertheless to be more difficult. Let us 
move to assumption~(b) and first remark that it is satisfied with high 
probability if $N$ is large compared to $2 \cdot \log_d p$. Thus, 
replacing eventually $N$ by $3 \cdot \log_d p$ (which does not affect 
the complexity), the assumption~(b) is harmless on average --- but maybe
not on particularly bad instances. We finally underline that, if we are
just interested in computing the $j$-th subresultant for a particular
$j$, then we just need to assume the non-vanishing of the scalar
subresultants in the range $[j{+}1, d{-}1]$.

Beyond this, one may wonder if one can use similar technics to compute 
not only subresultants but cofactors as well. For those indexes $j$ such 
that $r_j$ is invertible in $\A$, the same analysis applies almost 
\emph{verbatim}. However for other indexes $j$, the differential 
computation seems to be much more subtle. One can get around this issue 
by using lifting technics only when $r_j$ is invertible and tracking 
precision naively otherwise: it is possible to get this way a stable 
algorithm whose average running time is acceptable but which seems to be 
bad in the worst case. Can we do better?

Another quite interesting question is those of designing an algorithm 
which combines the precision technology developed in this paper with the 
``half-\textsc{gcd}'' methods. It is actually closely related to the 
previous question because ``half-\textsc{gcd}'' methods make an 
intensive use of cofactors in order to speed up the computation.

\section{Conclusion: p-adic floats?}

When computing with real numbers, computers very often use floating 
point arithmetic. 
The rough idea of this model consists in representating all real numbers 
using the same number of digits (the so-called \emph{precision}) and to
apply rounding heuristics when final digits are unsettled.
In comparison with arithmetic interval, floating point arithmetic has
two main advantages. First, it allows simple and fast implementations.
Second, experiments show the results obtained this way have generally
more much correct digits that those predicted by arithmetic interval.
The counterpart is that, expect on small examples, obtaining proved
results is generally intractable.

In the $p$-adic setting, the analogue of floating point arithmetic has 
not been developed yet. One reason for this is probably the well-known 
saying: ``in the $p$-adic world, rounding errors do not accumulate''. 
Consequently one might expect that interval arithmetic would provide 
sharp results. Nonetheless this hope is failing and examples are basic 
and numerous: $p$-adic differential equations \cite{bostan, padicdiff}, 
LU factorization \cite{LU}, SOMOS 4 sequence \cite{padicprec}, 
resultants (this paper), \emph{etc.} Consequently, interval arithmetic 
is not as good as one might have expected at first. Therefore, it 
probably makes sense to seriously study the analogue of floating point 
arithmetic in a ultrametric context.

Let us describe quickly what might be this analogue and what are its
advantages and disadvantages. We keep the notations of previous
sections: the letter $\A$ denotes a complete discrete valuation ring
with uniformizer $\pi$ and $K$ is its fraction field.
In the model of ultrametric floating point arithmetic, we fix a
positive integer $N$ (the \emph{precision}) and represent elements 
of $K$ by approximations of the form:
\begin{equation}
\label{eq:padicfloat}
\pi^e \cdot \sum_{i=0}^{N-1} x_i \pi^i
\end{equation}
where $e$ is a relative integer and the $x_i$'s are elements of a fixed 
set of representatives of $\A$ modulo $\pi$ with the convention that the 
representative of $0 \in k$ is $0 \in \A$.
We further assume that $x_0 \neq 0$, \emph{i.e.} $e$ is the valuation
of the sum \eqref{eq:padicfloat}. We see that this framework is quite
similar to usual floating point arithmetics: the integer $e$ plays the
role of exponent, the uniformizer $\pi$ plays the role of the basis
and the value $\sum_{i=0}^{N-1} x_i \pi^i$ plays the role of the
significand (the mantissa). It remains to define operations $\oplus$
and $\odot$ on approximations modeling addition and multiplication
on $K$ respectively. We do this as follows: given $x$ and $y$ two
elements of $K$ of the form Eq.~\eqref{eq:padicfloat}, we compute
$x+y$ (resp. $xy$) in $K$, expand it as a convergent series 
$\sum_{i=v}^{\infty} s_i \pi^i$ (with $s_v \neq 0$) and define $x 
\oplus y$ (resp. $x \odot y$) by truncating the series at $i = v+N$.

Similarly to real floating point arithmetic, the main advantages of 
ultrametric floating point arithmetic are the simplicity and the 
efficiency and the counterpart is the difficulty to get proved results. 
Moreover, the aforementioned examples are evidences that ultrametric 
floating point arithmetic may often compute much more correct digits 
than those predicted by an analysis based on interval arithmetic. In 
order to illustrate this last assertion, let us go back to the case of 
resultants discussed earlier in this paper. Let $A$ and $B$ be two monic 
polynomials of degree $d$ (picked at random) whose coefficients are all 
known at precision $O(\pi^N)$. We have proved that if we are using the
model of interval arithmetic, then the subresultant pseudo-remainder
sequence algorithm will output $\Res(A,B)$ at precision 
$O(\pi^{N-N_{\text{int}}})$ where $N_{\text{int}}$ grows linearly 
with respect to $d$ on average. On the other hand, if we are using 
ultrametric floating point arithmetic, then the same algorithm will
output $\Res(A,B)$ at precision $O(\pi^{N-N_{\text{float}}})$ where 
$N_{\text{float}}$ grows linearly with respect to $\log d$ on average.
We emphasize furthermore that this result is \emph{proved}! From this
point of view, floating point arithmetics seems to behave better in
the ultrametric setting: we may hope to get proved results relatively
cheaply.

\begin{thebibliography}{99}
\footnotesize
\bibitem{andrews}
  G.~Andrews,
  \emph{The Theory of Partitions},
  Cambridge University Press (1976)
\bibitem{pari}
  C.~Batut, K.~Belabas, D.~Benardi, H.~Cohen, M.~Olivier, 
  \emph{User’s guide to PARI-GP} (1985--2013)
\bibitem{bostan}
  A.~Bostan, L.~Gonz\'alez-Vega, H.~Perdry, É.~Schost, 
  \emph{From Newton sums to coefficients: complexity issues in characteristic $p$}, 
  MEGA’05 (2005)
\bibitem{magma}
  W.~Bosma, J.~Cannon, C.~Payoust, 
  \emph{The Magma algebra system. I. The user language.}
  J. Symbolic Comput. {\bf 24} (1997), 235--265
\bibitem{padicprec}
  X.~Caruso, D.~Roe, T.~Vaccon,
  \emph{Tracking $p$-adic precision},
  LMS J. Comp. and Math. {\bf 17}, 274--294
\bibitem{LU}
  X.~Caruso,
  \emph{Random matrices over a DVR and LU factorization},
  to appear at J. Symb. Comp.
\bibitem{cohen}
  H.~Cohen,
  \emph{A course in Computational Algebraic Number Theory},
  Springer (1996)
\bibitem{kedlaya}
  K.~Kedlaya,
  \emph{Counting points on hyperelliptic curves using Monsky--Washnitzer cohomology}, 
  J. Ramanujan Math. Soc. {\bf 16} (2001), 323--338
\bibitem{padicdiff}
  P.~Lairez, T.~Vaccon,
  \emph{Computation of power series solutions with $p$-adic coefficients of certain differential equations},
  preprint (2014)
\bibitem{sage}
  W.~Stein et al.
  \emph{Sage Mathematics Software}, 
  The Sage Development Team (2005--2013)
\bibitem{winkler}
  F.~Winkler,
  \emph{Polynomial Algorithms in Computer Algebra},
  Springer Wien New Work (1996)
\bibitem{}
\end{thebibliography}


\end{document}
