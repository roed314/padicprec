\documentclass[12pt,a4paper]{article}

%***************************************************************
% Choisissez l'encodage:
%\usepackage[latin1]{inputenc}            % pour Windows
\usepackage[utf8]{inputenc}                % pour Linux
%\usepackage[applemac]{inputenc}        % pour Mac
%****************************************************************

\usepackage[T1]{fontenc}
\usepackage[english,francais]{babel}
\usepackage{graphicx}
\usepackage{enumerate}
\usepackage{amsmath,amssymb,amsthm}

\newcommand{\Z}{\mathbb Z}
\newcommand{\E}{\mathbb E}
\newcommand{\prob}{\mathbb P}
\newcommand{\pgcd}{\text{\sc pgcd}}
\newcommand{\Res}{\text{\rm Rés}}

\newtheorem{theo}{Théorème}

\begin{document}

\thispagestyle{empty}  

\title{\textbf{Résultants et sous-résultants de polynômes $p$-adiques}}

\author{\textbf{X.~Caruso} \\
IRMAR \\
Université Rennes 1 \\
Campus de Beaulieu \\
35042 Rennes Cedex \\
\texttt{xavier.caruso@normalesup.org}}
\date{}

\maketitle

Soient $p$ un nombre premier et $\Z_p$ l'anneau des entiers 
$p$-adiques. À l'origine de ce travail est la volonté d'obtenir
des algorithmes à la fois efficaces et stables numériquement pour
le calcul du $\pgcd$ --- ainsi que des coefficients de Bézout ---
de polynômes à coefficients dans $\Z_p$.

\medskip

Un premier candidat est, bien entendu, l'algorithme d'Euclide usuel. 
Malheureusement, s'il est plutôt efficace, on s'aperçoit rapidement 
qu'il ne fait pas le poids au niveau de la stabilité numérique. En 
effet, on observe expérimentalement que sur des entrées aléatoires $A$ 
et $B$ piochées parmi les polynômes unitaires de degré $d$ à 
coefficients dans $\Z_p$, l'algorithme d'Euclide étendu calcule les 
coefficients de Bézout correspondants $U$ et $V$ avec une perte moyenne 
d'un nombre de chiffres significatifs sur chaque coefficient qui croît 
proportionnellement à $d$.
De surcroît, on obtient fréquemment des exemples pour lesquels on 
observe une chute importante de la précision alors que le résultant 
$\Res(A,B)$ est inversible dans $\Z_p$. Or, avec un algorithme stable,
ceci ne devrait pas se produire car la théorie des résultants affirme 
que les coefficients de $U$ et $V$ s'écrivent comme le quotient d'une 
expression polynômiale en les coefficient de $A$ et $B$ par $\Res(A,B)$ ; 
ainsi, si $\Res(A,B)$ est inversible dans $\Z_p$, on s'attend à 
connaître les coefficients de Bézout avec la même précision que les 
entrées.

\bigskip

La première partie de mon exposé sera consacrée à l'explication des 
phénomènes qui viennent d'être décrits. Plus précisément, je 
montrerai que les pertes de précision qui s'accumulent au cours de 
l'exécution de l'algorithme d'Euclide valent approximativement :
\begin{equation}
\label{eq:perteEuclide}
2 \cdot \sum_{j=0}^{d-1} V_j(A,B)
\end{equation}
où $V_j(A,B)$ désigne la valuation de coefficient dominant du $j$-ième
sous-résultant de $(A,B)$.
Cette quantité est à mettre en comparaison avec la valeur $2 \cdot 
V_0(A,B)$ qui est la perte \og théorique \fg\ donnée par l'argument
des résultants. J'étudierai ensuite les fonctions $V_j$ considérées 
comme des variables aléatoires : j'énoncerai un théorème qui décrit 
leur loi et, comme corollaire, en déduirai les résultats descriptifs 
que voici.

\begin{theo}
Pour tout $j \in \{0, \ldots, d-1\}$, on a :
\begin{enumerate}[i)]
\item $\frac 1 {p-1} \leq \E[V_j] \leq \frac p {(p-1)^2}$ ;
\item $\sigma(V_j) = O\big(\frac 1 {\sqrt p}\big)$ ;
\item $\prob[V_j \leq m] = O\big( p^{-m + O(\sqrt m)} \big)$
\end{enumerate}
où les constantes dans tous les $O(\cdot)$ sont absolues (et, en
particulier, ne dépendent ni de $j$, ni de $d$).
\end{theo}

Il résulte du théorème que l'expression \eqref{eq:perteEuclide} vaut en 
moyenne $\simeq \frac{2d}{p-1}$, en accord avec ce qui avait été 
observé initialement. En comparaison, la perte \og théorique \fg\ $2 
\cdot V_0(A,B)$ ne vaut en moyenne que $\simeq \frac 2 {p-1}$. En
d'autres termes, l'algorithme d'Euclide surestime les pertes de
précision d'un facteur $d$.

\bigskip

Enfin, dans une deuxième partie de mon exposé, je présenterai une 
variante \og stabilisée \fg\ de l'algorithme d'Euclide qui conserve sa 
complexité mais atteint également la perte de précision donnée par la 
théorie des résultants. Cette variante repose de façon essentielle sur
la théorie de la précision $p$-adique développée dans \cite{padicprec}.

\vspace{30pt}

\renewcommand{\section}[2]{}%
\textbf{Bibliographie}
\begin{thebibliography}{0}
\bibitem{padicprec}
{\sc X.~Caruso, D.~Roe, T.~Vaccon},
{\em Tracking $p$-adic precision},
LMS J. Comput. Math. {\bf 17} (Special issue A), 2014,  274--294
\end{thebibliography}


\end{document}
    
