\documentclass{beamer}
%\documentclass[handout]{beamer}

\usepackage[utf8]{inputenc}
\usepackage[english]{babel}
\usepackage{amsmath,amssymb}
\usepackage{color}
\usepackage{tikz}
\usepackage{times}
\usepackage{verbatim}
\usepackage{array}
\usepackage{multirow}
\usepackage{stmaryrd}
\usepackage{comment}

\definecolor{textcolor}{rgb}{0.1,0.1,0.2}
\definecolor{alert}{rgb}{0.7,0.1,0.1}
\definecolor{bleu}{rgb}{0.1,0.1,0.5}
\definecolor{violet}{rgb}{0.5,0.1,0.5}

\definecolor{bleufonce}{rgb}{0,0,0.4}
\definecolor{rougefonce}{rgb}{0.4,0,0}
\definecolor{vertfonce}{rgb}{0.1,0.4,0.1}
\definecolor{beige}{rgb}{0.9,0.9,0.7}


\setbeamercolor{background canvas}{bg=white}
\setbeamercolor{normal text}{fg=textcolor}
\setbeamercolor{structure}{fg=textcolor}
\setbeamercolor{frametitle}{bg=white,fg=black}
\setbeamertemplate{navigation symbols}{}%remove navigation symbols

\setbeamercolor{postit}{fg=vertfonce,bg=yellow!50}

\newtheorem{prop}{Proposition}

\definecolor{bleu}{rgb}{0,0,0.6}
\definecolor{bleufonce}{rgb}{0,0,0.4}
\definecolor{rouge}{rgb}{0.7,0.2,0.2}
\definecolor{vert}{rgb}{0.1,0.3,0.1}
\definecolor{vert2}{rgb}{0.2,0.4,0.2}
\definecolor{violet}{rgb}{0.5,0,0.5}
\definecolor{jaune}{rgb}{0.4,0.4,0.2}

\newcommand\Wider[2][3em]{%
\makebox[\linewidth][c]{%
  \begin{minipage}{\dimexpr\textwidth+#1\relax}
  \raggedright#2
  \end{minipage}%
  }%
}

\newenvironment<>{varblock}[2][\textwidth]{
    \begin{center}
      \begin{minipage}{#1}
        \setlength{\textwidth}{#1}
        \setbeamercolor{block title}{fg=white,bg=jaune}
          \begin{actionenv}#3
            \def\insertblocktitle{#2}
            \par
            \usebeamertemplate{block begin}}
  {\par
      \usebeamertemplate{block end}
    \end{actionenv}
  \end{minipage}
\end{center}}

\newcommand{\ph}{\vphantom{$A^A_A$}}
\newcommand{\Z}{\mathbb Z}
\newcommand{\Q}{\mathbb Q}
\newcommand{\Zp}{\Z_p}
\newcommand{\Qp}{\Q_p}
\newcommand{\R}{\mathbb R}
\newcommand{\C}{\mathbb C}

\begin{document}

\title{p-Adic Stability in Linear Algebra}
\author{Xavier Caruso, David Roe, Tristan Vaccon}
\date{July 6, 2015}


\begin{frame}[plain]
\setbeamercolor{titre}{fg=bleu,bg=bleu!10}

\hfill
\begin{beamercolorbox}[sep=1em,wd=7.5cm,rounded=true]{titre}
\large \bf \centering
p-Adic Stability

in Linear Algebra
\end{beamercolorbox}
\hfill \null

\hfill
{\footnotesize
\begin{tabular}{>{\centering}p{3cm}>{\centering}p{3.5cm}>{\centering}p{3.5cm}l}
Xavier Caruso & David Roe & Tristan Vaccon & \\
Université Rennes 1 & Univ. British Columbia &
Université Rennes 1 & \vspace{-0.1cm} \\
%\texttt
{\tiny xavier.caruso@normalesup.org} &
%\texttt
{\tiny roed.math@gmail.com} &
%\texttt
{\tiny tristan.vaccon@univ-rennes1.fr} &
\end{tabular}
}
\hfill \null

\vspace{0.2cm}

{\color{bleu} \hfill \hrulefill \hfill \null}

\vspace{0.5cm}

\hfill
{\small \color{vert}
{\bf I}nternational 
{\bf S}ymposium on%
}
\hfill \null

\hfill
{\small \color{vert}
{\bf S}ymbolic 
and {\bf A}lgebraic
{\bf C}omputation%
}
\hfill \null

\vspace{0.5cm}

\hfill
July 6, 2015
\hfill \null

\end{frame}

\begin{frame}
\frametitle{What are $p$-adic numbers?} \pause

{\color{alert} $p$} refers to a prime number

\bigskip \pause

{\color{alert} $p$-adic numbers} are numbers written in $p$-basis
of the shape:

\vspace{-0.3cm}

$$a = \ldots a_i \ldots a_2 \: a_1 \: a_0 \, , \, a_{-1} \: a_{-2} \ldots 
\: a_{-n}$$

\vspace{-0.1cm}

with $0 \leq a_i < p$ for all $i$.

\medskip \pause

Addition and multiplication on these numbers are defined by
applying SchoolBook algorithms.

\medskip \pause

The {\color{alert} valuation} $v_p(a)$ of $a$ is the smallest $v$
such that $a_v \neq 0$.

\medskip \pause

The $p$-adic numbers form the field {\color{alert} $\Qp$}.

\bigskip \pause

A $p$-adic number with no digit after the comma is a

{\color{alert} $p$-adic integer}.

\medskip \pause

The $p$-adic integers form a subring {\color{alert} $\Zp$} of $\Qp$.
\end{frame}

\begin{frame}
\frametitle{Why should we use/study them?} \pause

\Wider[1cm]{
\begin{itemize}
\item they are already ubiquitous in number theory

\smallskip \pause

{\scriptsize
Some interesting questions (coming from number theory) involve
$p$-adic numbers

\emph{e.g.} local-global principle, $p$-adic Galois representations

\smallskip

$p$-adic cohomologies (on which Kedlaya-type counting points algorithms 
are based)
} 

\medskip \pause

\item they allow to define more functions in positive characteristic

\smallskip \pause

{\scriptsize
{\bf Typical naive example:}

they can be used to give a sense to
the function $\frac{x^p - x} p$ over a field of char. $p$.

\smallskip \pause

Related ideas were used to 
\begin{enumerate}[$\bullet$] \scriptsize
\item \scriptsize compute composed sums/products of polynomials \pause
\item \scriptsize compute isogenies between elliptic curves
\end{enumerate}
}

\medskip \pause

\item they sometimes offer good stability (compared to $\R$)

\smallskip \pause

{\scriptsize
We start with a problem defined over $\Z$,
we solve it in $\Zp$ and do reconstruction.

\smallskip \pause

These ideas were notably widely used in linear algebra

\vspace{-0.1cm} \pause

\emph{e.g.} inversion of the Hilbert matrix $(\frac 1{i+j-1})_{1 \leq i,j 
\leq n}$
}

\medskip \pause

\item and many other reasons...

\end{itemize}}
\end{frame}

\begin{frame}
\frametitle{$\Zp$ on a computer} \pause

A $p$-adic integer carries an infinite amount of information.

\medskip \pause

Therefore it cannot fit in the memory of a computer and we need to
work with \emph{truncated} $p$-adic integers\pause,

\emph{i.e.} we only specify the $N$ last digits $a_0, \ldots, a_{N-1}$.

\bigskip \pause

These truncated $p$-adic integers are often denoted by:

\vspace{-0.3cm}

$$a_{N-1} \ldots a_2 \: a_1 \: a_0 + O(p^N)
\quad \text{or} \quad
\ldots a_{N-1} \ldots a_2 \: a_1 \: a_0.$$ \pause

\vspace{-0.3cm}

The {\color{alert} absolute precision} is $N$.

\smallskip \pause

The {\color{alert} relative precision} is $N - v_p(a) \pause = \text{
\# significant digits}$.

\bigskip \pause

{\bf Arithmetic operations on truncations}

\vspace{-0.5cm} \pause

$$\begin{array}{l}
\big( a + O(p^{N_a}) \big) + \big( b + O(p^{N_b}) \big) =
a + b + O(p^{\min(N_a, N_b)}) \medskip \pause \\
\big( a + O(p^{N_a}) \big) \times \big( b + O(p^{N_b}) \big) 
= ab + O(p^{\min(N_a + v_p(b), N_b + v_p(a))})
\end{array}$$
\end{frame}


\begin{frame}
\frametitle{But... \only<2->{studying precision is actually not quite trivial}} 
\pause \pause

{\bf Question:} Compute the determinant of
$$\left(
\raisebox{-0.95cm}{
\begin{tikzpicture}[xscale=3, yscale=0.8]

\only<3-5>{
  \node at (0,0) { $\ldots1014240$ };
}
\only<6-8>{
  \node at (0,0) { \color{alert} $\ldots1014240$ };
}
\only<3-8>{
  \node at (1,0) { $\ldots4324032$ };
  \node at (2,0) { $\ldots0101111$ };
}
\only<3-6>{
  \node at (0,-1) { $\ldots4132310$ };
  \node at (1,-1) { $\ldots4020033$ };
  \node at (2,-1) { $\ldots4421144$ };
}
\only<7-8>{
  \node at (0,-1) { \color{black!30} $\ldots0000000$ };
  \node at (1,-1) { $\ldots3414440$ };
  \node at (2,-1) { $\ldots4223420$ };
}
\only<3-7>{
  \node at (0,-2) { $\ldots2202130$ };
  \node at (1,-2) { $\ldots2220114$ };
  \node at (2,-2) { $\ldots4204122$ };
}
\only<8>{
  \node at (0,-2) { \color{black!30} $\ldots0000000$ };
  \node at (1,-2) { $\ldots0224020$ };
  \node at (2,-2) { $\ldots2313110$ };
}
\only<9-13>{
  \node at (0,0) { $\ldots1014240$ };
  \node at (1,0) { $\ldots4324032$ };
  \node at (2,0) { $\ldots0101111$ };
  \node at (0,-1) { \color{black!30} $\ldots0000000$ };
  \node at (0,-2) { \color{black!30} $\ldots0000000$ };
  \node at (2,-1) { $\ldots4223420$ };
}
\only<9,12-13>{
  \node at (1,-1) { $\ldots3414440$ };
}
\only<10-11>{
  \node at (1,-1) { \color{alert} $\ldots3414440$ };
}
\only<9-10>{
  \node at (1,-2) { $\ldots0224020$ };
  \node at (2,-2) { $\ldots2313110$ };
}
\only<11-13>{
  \node at (1,-2) { \color{black!30} $\ldots0000000$ };
  \node at (2,-2) { $\ldots3014000$ };
}

\only<14,24->{
  \node at (0,0) { $\ldots1014240$ };
  \node at (1,0) { $\ldots4324032$ };
  \node at (2,0) { $\ldots0101111$ };
  \node at (0,-1) { $\ldots4132310$ };
  \node at (1,-1) { $\ldots4020033$ };
  \node at (2,-1) { $\ldots4421144$ };
  \node at (0,-2) { $\ldots2202130$ };
  \node at (1,-2) { $\ldots2220114$ };
  \node at (2,-2) { $\ldots4204122$ };
}
\only<15>{
  \node at (0,0) { $\ldots4324032$ };
}
\only<16-18>{
  \node at (0,0) { \color{alert} $\ldots4324032$ };
}
\only<15-23>{
  \node at (1,0) { $\ldots1014240$ };
  \node at (2,0) { $\ldots0101111$ };
}
\only<15-16>{
  \node at (0,-1) { $\ldots4020033$ };
  \node at (1,-1) { $\ldots4132310$ };
  \node at (2,-1) { $\ldots4421144$ };
}
\only<17-18>{
  \node at (0,-1) { \color{black!30} $\ldots0000000$ };
  \node at (1,-1) { $\ldots4043200$ };
  \node at (2,-1) { $\ldots1221300$ };
}
\only<15-17>{
  \node at (0,-2) { $\ldots2220114$ };
  \node at (1,-2) { $\ldots2202130$ };
  \node at (2,-2) { $\ldots4204122$ };
}
\only<18>{
  \node at (0,-2) { \color{black!30} $\ldots0000000$ };
  \node at (1,-2) { $\ldots2223100$ };
  \node at (2,-2) { $\ldots1100400$ };
}
\only<19-23>{
  \node at (0,0) { $\ldots4324032$ };
  \node at (1,0) { $\ldots1014240$ };
  \node at (2,0) { $\ldots0101111$ };
  \node at (0,-1) { \color{black!30} $\ldots0000000$ };
  \node at (0,-2) { \color{black!30} $\ldots0000000$ };
  \node at (2,-1) { $\ldots1221300$ };
}
\only<19,22-23>{
  \node at (1,-1) { $\ldots4043200$ };
}
\only<20-21>{
  \node at (1,-1) { \color{alert} $\ldots4043200$ };
}
\only<19-20>{
  \node at (1,-2) { $\ldots2223100$ };
  \node at (2,-2) { $\ldots1100400$ };
}
\only<21-23>{
  \node at (1,-2) { \color{black!30} $\ldots0000000$ };
  \node at (2,-2) { $\ldots3014000$ };
}

\end{tikzpicture}
}\right)$$

\medskip \pause

{\bf Answer:}

\begin{center}
\begin{tabular}{rr}
just expand: & $\ldots4400000$\medskip \\
\uncover<5->{Hermite + expand:} & \uncover<13->{$\ldots34400000$}\medskip \\
\uncover<14->{Smith + expand:} & \uncover<23->{$\ldots234400000$}
\end{tabular}
\end{center}

\uncover<24->{\color{violet} \bf What is the correct precision?}

\end{frame}

\begin{frame}
\begin{center}
\bf \huge
The precision Lemma
\end{center}
\end{frame}

\begin{frame}
\frametitle{Lattices} \pause

Let $E$ be a finite dimensional $\Qp$-vector space.

\medskip \pause

A {\color{alert} lattice} $H$ of $E$ is a sub-$\Zp$-module of 
$E$ generated by a $\Qp$-basis of $E$.

\bigskip \pause

\begin{overlayarea}{\textwidth}{9cm}
{\bf Geometrical interpretation}

\medskip \pause

\only<5->{
  Recall that $\Qp$ is endowed with the 

  ultrametric norm $|x| = p^{-v_p(x)}$
}

\medskip

\only<6->{
  Endow $E$ with a norm
}

\only<7->{
  \vspace{-1mm}

  {\scriptsize (compatible with that on $\Qp$)}
}

\medskip

\only<9->{
The unit ball $B_E(1)$ is a lattice
}

\smallskip

\only<11->{
{\scriptsize
and other lattices are deduced from it by

\vspace{-1mm}

applying a linear isomorphism
$\varphi : E \stackrel{\sim}{\longrightarrow} E$.
}}

\medskip

\only<13->{
Lattices might then be thought as 

special neighborhoods of $0$.
}
\end{overlayarea}

\vspace{-9.6cm}

\hfill
\begin{tikzpicture}
\draw[transparent] (0,0) rectangle (4,6);

\only<8->{
\begin{scope}[beige, rounded corners=10pt]
\fill (0.1,0.2) rectangle (3.9,5.8);
\end{scope}

\only<10-11>{
  \draw[black,fill=green!50] (1,2) rectangle (3,4);
  \fill[vertfonce] (2,3) circle (0.5mm);
  \node[vertfonce,scale=0.7] at (2.15,2.9) { $0$ };
  \node[vertfonce,scale=0.8] at (2,1.7) { $B_E(1)$ };
}
\only<12->{
  \begin{scope}[xshift=2cm,yshift=3cm,rotate=20]
  \draw[black,fill=green!50] (-0.7,-1.5) rectangle (0.7,1.5);
  \draw[black,->] (-1,0)--(1,0);
  \draw[black,->] (0,-2)--(0,2);
  \end{scope}
}
\begin{scope}[textcolor,very thick, rounded corners=10pt]
\draw (0.1,0.2) rectangle (3.9,5.8);
\end{scope}}
\end{tikzpicture}
\end{frame}

\begin{frame}
\frametitle{Statement of the Lemma} \pause

We consider:

\medskip

\begin{tabular}{rl}
$E, F$ & two finite dimensional $\Qp$-vector spaces, \\
$U$ & an open subset in $E$, \\
$f$ & a differentiable function $U \to F$.
\end{tabular}

\medskip \pause

Let $x \in U$.
\emph{We assume that $f'(x) : E \to F$ is surjective.}
\medskip \pause

Then, for all lattice $H \subset E$ ``sufficiently small'':

\vspace{-0.2cm}

$$f(x+H) = f(x) + f'(x)(H).$$

\bigskip \pause

{\color{vertfonce}
The correct meaning of ``sufficiently small'' is rather subtle.

\smallskip \pause

Nevertheless, it can be made completely explicit...

at least if $f$ has more regularity.
}

\end{frame}


\begin{frame}
\frametitle{Illustration} \pause

\Wider{
\begin{tikzpicture}
\draw[transparent] (0,0.4) rectangle (12,8);

\begin{scope}[beige, rounded corners=10pt]
\fill (0.1,1) rectangle (3.9,7.8);
\fill (4.1,1) rectangle (7.9,7.8);
\fill (8.1,1) rectangle (11.9,7.8);
\end{scope}

\begin{scope}[thick]
\draw[->] (2.5,0.9) to[out=-25,in=-155] (5.5,0.9);
\node at (4,0.75) { $f_1$ };
\draw[->] (6.5,0.9) to[out=-25,in=-155] (9.5,0.9);
\node at (8,0.75) { $f_2$ };
\end{scope}

\only<3->{
  \draw[black,fill=green!50] (1.2,3.7) rectangle (2.8,5.3);
  \node[vertfonce,scale=0.8] at (2,3.4) { $x + H$ };
  \fill[vertfonce] (2,4.5) circle (0.5mm);
  \node[vertfonce,scale=0.7] at (2.15,4.4) { $x$ };
}
\only<4->{
  \draw[black,fill=green!50,xshift=6cm,yshift=4.5cm,rotate=50]
    (-0.8,-1.5) rectangle (0.8,1.5);
  \node[vertfonce,scale=0.8] at (6,2.6) { $f_1(x) + f'_1(x)(H)$ };
  \fill[vertfonce] (6.5,4.4) circle (0.5mm);
  \node[vertfonce,scale=0.7] at (6.5,4.15) { $f_1(x)$ };
}
\only<5->{
  \draw[black,fill=green!50,xshift=10cm,yshift=4.5cm,rotate=-20]
    (-0.5,-2) rectangle (0.5,2);
  \node[vertfonce,scale=0.8] at (10,2.2) { $f(x) + f'(x)(H)$ };
  \node[vertfonce,scale=0.6] at (10,1.85) { with $f = f_2 \circ f_1$ };
  \fill[vertfonce] (10.6,6) circle (0.5mm);
  \node[vertfonce,scale=0.7] at (10.5,5.75) { $f(x)$ };
}

\begin{scope}[textcolor,very thick, rounded corners=10pt]
\draw (0.1,1) rectangle (3.9,7.8);
\draw (4.1,1) rectangle (7.9,7.8);
\draw (8.1,1) rectangle (11.9,7.8);
\end{scope}
\end{tikzpicture}
}
\end{frame}

\begin{frame}
\frametitle{One application: determinant} \pause

\begin{overlayarea}{\textwidth}{9cm}
{\bf Question:} Compute the determinant of
$$\left(
\raisebox{-0.95cm}{
\begin{tikzpicture}[xscale=3, yscale=0.8]
\node at (0,0) { $\ldots1014240$ };
\node at (1,0) { $\ldots4324032$ };
\node at (2,0) { $\ldots0101111$ };
\node at (0,-1) { $\ldots4132310$ };
\node at (1,-1) { $\ldots4020033$ };
\node at (2,-1) { $\ldots4421144$ };
\node at (0,-2) { $\ldots2202130$ };
\node at (1,-2) { $\ldots2220114$ };
\node at (2,-2) { $\ldots4204122$ };
\end{tikzpicture}
}\right)$$

\medskip

{\bf Answer:}

\begin{center}
\begin{tabular}{rr}
just expand: & $\ldots4400000$\medskip \\
Hermite + expand: & $\ldots34400000$\medskip \\
Smith + expand: & $\ldots234400000$
\end{tabular}
\end{center}

{\color{violet} \bf What is the correct precision?}
\end{overlayarea}

\vspace{-9.2cm} \pause

\begin{tikzpicture}
\fill[white,opacity=0.95] (0,0) rectangle (10,9.2);
\end{tikzpicture}

\vspace{-9.2cm} 

Consider the function $\det : M_n(\Qp) \to \Qp$.

\medskip \pause

Given $M \in M_n(\Qp)$, it is well known that:

\vspace{-0.2cm}

$$\begin{array}{rcl}
det'(M) : \quad M_n(\Qp) & \to & \Qp \\
dM & \mapsto & \text{Tr}(\text{Com}(M) \cdot dM)
\end{array}$$

\smallskip \pause

{\bf Proposition}

\smallskip \pause

Assume that $M$ is invertible and 
write its Smith decomposition:

\vspace{-0.2cm}

$$\begin{array}{rl}
M = P \cdot \text{Diag}(p^{n_1}, \cdots, p^{n_d}) \cdot Q, &
P, Q \in \text{GL}_n(\Zp) \\
& n_1 \leq \cdots \leq n_d
\end{array}$$

\vspace{-0.1cm} \pause

The image of $H_0 = M_n(\Zp)$ under $\det'(M)$ is
$p^{n_1 + \cdots + n_{d-1}} \Zp$.

\bigskip \pause

{\bf Corollary}

\smallskip \pause

Using Smith decomposition for computing determinant gives the
optimal precision.
\end{frame}

\begin{frame}
\begin{center}
\bf \huge 
The art of 

tracking precision
\end{center}
\end{frame}

\begin{frame}
\frametitle{Lattices as precision data} \pause

\hfill
\begin{beamercolorbox}[sep=0.3em,wd=5cm,rounded=true]{postit}
\centering
$f(x+H) = f(x) + f'(x)(H)$
\end{beamercolorbox}

\vspace{-0.8cm} \pause

Our proposal was to use 

lattices to encode precision

\smallskip \pause

{\footnotesize
Given a set of $d$ variables $x_1, \ldots, x_d \in \Qp$ 
(\emph{e.g.} a matrix),

replace $x_1 + O(p^{N_1}), \ldots, x_d + O(p^{N_d})$
by $(x_1, \ldots, x_d) + O(H)$
}

\medskip \pause

{\small \color{violet}
{\bf Advantage:} sharp on precision \pause

{\bf Drawback:} time/space consuming
}

\medskip \pause

This encompasses the usual way

of representing precision

\smallskip \pause

{\footnotesize
\begin{tabular}{r@{\,\,}l}
& $x_1 + O(p^{N_1}), \ldots, x_d + O(p^{N_d})$ \\
= & {\color{alert} diagonal lattice} 
$\left< p^{N_1} x_1^\star, \ldots, p^{N_d} x_d^\star \right>$.
\end{tabular}
}

\bigskip \pause

{\bf Definition.}
The {\color{alert} number of diffused 

digits} of a lattice $H$ is the length of

$H_{\text{diag}}/H$ where $H_{\text{diag}}$ is the minimal 

\emph{diagonal} 
lattice containing $H$.

\vspace{-5.4cm}

\hfill
\begin{tikzpicture}
\draw[transparent] (0,0) rectangle (4.5,5.5);

\only<10->{
\begin{scope}[beige, rounded corners=10pt]
\fill (0.2,0.2) rectangle (4.3,5.3);
\end{scope}

\begin{scope}[xshift=2.25cm,yshift=2.75cm,rotate=35]
\draw[black,fill=green!50] (-0.7,-1.5) rectangle (0.7,1.5);
\end{scope}

\only<11->{
  \begin{scope}[xshift=2.25cm,yshift=2.75cm]
  \draw[black,fill=red!50,opacity=0.8] (-1.44,-1.64) rectangle (1.44,1.64);
  \end{scope}
}
\begin{scope}[textcolor,very thick, rounded corners=10pt]
\draw (0.2,0.2) rectangle (4.3,5.3);
\end{scope}}
\end{tikzpicture}
\end{frame}

\begin{frame}
\frametitle{An example: product of matrices} \pause

\hfill
\begin{tikzpicture}
\begin{scope}
\clip[rounded corners=10pt] (0,0) rectangle (10,-3.4);
\fill[green!30] (0,0) rectangle (10,-3.4);
\fill[vertfonce] (0,0) rectangle (10,-1.1);
\end{scope}
\draw[very thick,rounded corners=10pt] (0,0) rectangle (10,-3.4);
\begin{scope}[color=white]
\node[right] at (0.1,-0.32) { \ph \bf Input: };
\node[right] at (1.6,-0.32) { \ph $A_1, \ldots, A_n \in M_2(\Zp)$ known at precision $O(p^N)$ };
\node[right] at (0.1,-0.8) { \ph \bf Output: };
\node[right] at (1.6,-0.8) { \ph the product $A_1 A_2 \cdots A_n$ };
\end{scope}
\node[right,color=vertfonce] at (0.1, -1.5) { \ph 1. };
\node[right] at (0.6, -1.5) { \ph $A = \text{identity matrix of size $2$ at precision } O(p^N)$ };
\node[right,color=vertfonce] at (0.1, -2) { \ph 2. };
\node[right] at (0.6, -2) { \ph {\color{bleufonce} for} $i$ in $1, 2, \ldots n$: };
\node[right,color=vertfonce] at (0.1, -2.5) { \ph 3. };
\node[right] at (1.1, -2.5) { \ph $A = A \cdot A_i$ };
\node[right,color=vertfonce] at (0.1, -3) { \ph 4. };
\node[right] at (0.6, -3) { \ph {\color{rougefonce} return} $A$ };
\end{tikzpicture}
\hfill \null

\vfill \pause 

\begin{overlayarea}{\textwidth}{4.2cm}

\hfill
\begin{tikzpicture}
\draw[transparent] (0,-0.62) rectangle (10,3.5);
\begin{scope}[beige, rounded corners=10pt]
\fill (0.2,0.2) rectangle (4.3,3.3);
\fill (5.7,0.2) rectangle (9.8,3.3);
\end{scope}
\begin{scope}[textcolor,very thick,rounded corners=10pt]
\draw (0.2,0.2) rectangle (4.3,3.3);
\draw (5.7,0.2) rectangle (9.8,3.3);
\end{scope}
\draw[very thick,->] (4.5,1.75)--(5.5,1.75)
  node[above, midway] { $\times A_i$ };
\node[brown, below right, scale=0.5] at (0.3,3.2) { $M_2(\Qp)$ };
\node[brown, below right, scale=0.5] at (5.8,3.2) { $M_2(\Qp)$ };
\only<4->{
  \begin{scope}[xshift=2.25cm,yshift=1.75cm]
  \draw[black,fill=green!50,opacity=0.8] (-1,-1) rectangle (1,1);
  \end{scope}
}
\only<5-6>{
  \begin{scope}[xshift=7.75cm,yshift=1.75cm]
  \draw[black,fill=green!50,opacity=0.8] (-1,-1) rectangle (1,1);
  \end{scope}
}
\only<6>{
  \node[left] at (10,-0.3) { no diffused digits };
}
\only<8-10>{
  \begin{scope}[xshift=7.75cm,yshift=1.75cm]
  \draw[black,fill=green!50,opacity=0.8] 
    (-1,-0.8)--(-0.8,1)--(1,0.8)--(0.8,-1)--cycle;
  \end{scope}
}
\only<9-10>{
  \begin{scope}[xshift=7.75cm,yshift=1.75cm]
  \draw[black,fill=red!50,opacity=0.8] (-1,-1) rectangle (1,1);
  \end{scope}
}
\only<10>{
  \node[left] at (10,-0.3) { \color{violet} diffused digits };
}
\end{tikzpicture}
\hfill \null

\end{overlayarea}
\end{frame}

\begin{frame}
\frametitle{An example: product of matrices}

\hfill
\begin{tikzpicture}
\begin{scope}
\clip[rounded corners=10pt] (0,0) rectangle (10,-3.4);
\fill[green!30] (0,0) rectangle (10,-3.4);
\fill[vertfonce] (0,0) rectangle (10,-1.1);
\end{scope}
\draw[very thick,rounded corners=10pt] (0,0) rectangle (10,-3.4);
\begin{scope}[color=white]
\node[right] at (0.1,-0.32) { \ph \bf Input: };
\node[right] at (1.6,-0.32) { \ph $A_1, \ldots, A_n \in M_2(\Zp)$ known at precision $O(p^N)$ };
\node[right] at (0.1,-0.8) { \ph \bf Output: };
\node[right] at (1.6,-0.8) { \ph the product $A_1 A_2 \cdots A_n$ };
\end{scope}
\node[right,color=vertfonce] at (0.1, -1.5) { \ph 1. };
\node[right] at (0.6, -1.5) { \ph $A = \text{identity matrix of size $2$ at precision } O(p^N)$ };
\node[right,color=vertfonce] at (0.1, -2) { \ph 2. };
\node[right] at (0.6, -2) { \ph {\color{bleufonce} for} $i$ in $1, 2, \ldots n$: };
\node[right,color=vertfonce] at (0.1, -2.5) { \ph 3. };
\node[right] at (1.1, -2.5) { \ph $A = A \cdot A_i$ };
\node[right,color=vertfonce] at (0.1, -3) { \ph 4. };
\node[right] at (0.6, -3) { \ph {\color{rougefonce} return} $A$ };
\end{tikzpicture}
\hfill \null

\vfill \pause 

\begin{overlayarea}{\textwidth}{4.2cm}

\hfill
\begin{tikzpicture}
\draw[transparent] (0,-0.62) rectangle (10,3.5);
\begin{scope}[beige, rounded corners=10pt]
\fill (0.2,0.2) rectangle (4.3,3.3);
\fill (5.7,0.2) rectangle (9.8,3.3);
\end{scope}
\begin{scope}[textcolor,very thick,rounded corners=10pt]
\draw (0.2,0.2) rectangle (4.3,3.3);
\draw (5.7,0.2) rectangle (9.8,3.3);
\end{scope}
\node[brown, below right, scale=0.5] at (0.3,3.2) { $M_2(\Qp)$ };
\node[brown, below right, scale=0.5] at (5.8,3.2) { $M_2(\Qp)$ };
\node[scale=0.8] at (2.25,-0.1) { without lattice };
\node[scale=0.8] at (7.75,-0.1) { with lattice };
\draw[black,dotted,xshift=2.25cm,yshift=1.75cm]
     (-1,-1) rectangle (1,1);
\draw[black,dotted,xshift=7.75cm,yshift=1.75cm]
     (-1,-1) rectangle (1,1);
\only<2>{\node at (5,-0.2) { $i = 0$ };}
\only<3>{\node at (5,-0.2) { $i = 1$ };}
\only<4>{\node at (5,-0.2) { $i = 2$ };}
\only<2-4>{
  \draw[black,fill=green!50,opacity=0.8,xshift=7.75cm,yshift=1.75cm] 
     (-1,-1) rectangle (1,1);
  \draw[black,fill=green!50,opacity=0.8,xshift=2.25cm,yshift=1.75cm]
     (-1,-1) rectangle (1,1);
}
\only<5-6>{
  \node at (5,-0.2) { $i = 3$ };
  \draw[black,fill=green!50,opacity=0.8,xshift=7.75cm,yshift=1.75cm]
    (-1,-0.8)--(-0.8,1)--(1,0.8)--(0.8,-1)--cycle;
}
\only<5>{
  \draw[black,fill=green!50,opacity=0.8,xshift=2.25cm,yshift=1.75cm]
    (-1,-0.8)--(-0.8,1)--(1,0.8)--(0.8,-1)--cycle;
}
\only<6-7>{
  \draw[black,fill=green!50,opacity=0.8,xshift=2.25cm,yshift=1.75cm]
     (-1,-1) rectangle (1,1);
}
\only<7>{
  \node at (5,-0.2) { $i = 4$ };
  \draw[black,fill=green!50,opacity=0.8,xshift=7.75cm,yshift=1.75cm]
    (-1,0.8)--(-0.8,-1)--(1,-0.8)--(0.8,1)--cycle;
}
\only<8-9>{
  \node at (5,-0.2) { $i = 5$ };
  \draw[black,fill=green!50,opacity=0.8,xshift=7.75cm,yshift=1.75cm,rotate=-20]
    (-0.7,-0.6) rectangle (0.7,0.6);
}
\only<8>{
  \draw[black,fill=green!50,opacity=0.8,xshift=2.25cm,yshift=1.75cm,rotate=10]
    (-0.9,-0.7) rectangle (0.9,0.7);
}
\only<9>{
  \draw[black,fill=green!50,opacity=0.8,xshift=2.25cm,yshift=1.75cm]
    (-1,-0.8) rectangle (1,0.8);
}
\only<10>{
  \node at (5,-0.2) { $i = 10$ };
  \draw[black,fill=green!50,opacity=0.8,xshift=7.75cm,yshift=1.75cm,rotate=17]
    (-0.6,-0.6) rectangle (0.6,0.6);
  \draw[black,fill=green!50,opacity=0.8,xshift=2.25cm,yshift=1.75cm]
    (-0.9,-0.8) rectangle (0.9,0.8);
}
\only<11>{
  \node at (5,-0.2) { $i = 100$ };
  \draw[black,fill=green!50,opacity=0.8,xshift=7.75cm,yshift=1.75cm,rotate=45]
    (-0.2,-0.3) rectangle (0.2,0.3);
  \draw[black,fill=green!50,opacity=0.8,xshift=2.25cm,yshift=1.75cm]
    (-0.7,-0.65) rectangle (0.7,0.65);
}
\end{tikzpicture}
\hfill \null

\end{overlayarea}
\end{frame}

\begin{frame}
\frametitle{An example: product of matrices}

\hfill
\begin{tikzpicture}
\begin{scope}
\clip[rounded corners=10pt] (0,0) rectangle (10,-3.4);
\fill[green!30] (0,0) rectangle (10,-3.4);
\fill[vertfonce] (0,0) rectangle (10,-1.1);
\end{scope}
\draw[very thick,rounded corners=10pt] (0,0) rectangle (10,-3.4);
\begin{scope}[color=white]
\node[right] at (0.1,-0.32) { \ph \bf Input: };
\node[right] at (1.6,-0.32) { \ph $A_1, \ldots, A_n \in M_2(\Zp)$ known at precision $O(p^N)$ };
\node[right] at (0.1,-0.8) { \ph \bf Output: };
\node[right] at (1.6,-0.8) { \ph the product $A_1 A_2 \cdots A_n$ };
\end{scope}
\node[right,color=vertfonce] at (0.1, -1.5) { \ph 1. };
\node[right] at (0.6, -1.5) { \ph $A = \text{identity matrix of size $2$ at precision } O(p^N)$ };
\node[right,color=vertfonce] at (0.1, -2) { \ph 2. };
\node[right] at (0.6, -2) { \ph {\color{bleufonce} for} $i$ in $1, 2, \ldots n$: };
\node[right,color=vertfonce] at (0.1, -2.5) { \ph 3. };
\node[right] at (1.1, -2.5) { \ph $A = A \cdot A_i$ };
\node[right,color=vertfonce] at (0.1, -3) { \ph 4. };
\node[right] at (0.6, -3) { \ph {\color{rougefonce} return} $A$ };
\end{tikzpicture}
\hfill \null

\vfill \pause

\begin{overlayarea}{\textwidth}{4.2cm}

\vspace{0mm}

\begin{center}
\renewcommand{\arraystretch}{1.2}
\begin{tabular}{|c|c|c|}
\hline
\multirow{2}{*}{\hspace{0.2cm}$n$\hspace{0.2cm}} &
\multicolumn{2}{c|}{Average 
\makebox[2.7cm]{\only<2>{gain of absolute}\only<3>{loss\,\,of\,\,\,{\color{alert} relative}}}
precision on $A_{1,1}$} \\
\cline{2-3}
& \hspace{0.4cm} without lattice \hspace{0.4cm} & with lattice \\
\hline
$10$ & 
\only<2>{\hphantom{00}1.2}\only<3>{$\hphantom{00}2.8$} & 
\only<2>{\hphantom{00}1.6}\only<3>{$\hphantom{00}2.4$} \\
$100$ & 
\only<2>{\hphantom{0}16.7}\only<3>{$\hphantom{0}17.1$} & 
\only<2>{\hphantom{0}28.7}\only<3>{$\hphantom{00}5.1$} \\
$1000$ &
\only<2>{$175.4$}\only<3>{$158.2$} & 
\only<2>{$325.7$}\only<3>{$\hphantom{00}7.9$} \\
\hline
\end{tabular}

\medskip

{\footnotesize
Results for a sample of $1000$ random inputs in $M_2(\Z_2)^n$}
\end{center}

\end{overlayarea}
\end{frame}

\begin{frame}
\frametitle{Our paper also studies...} \pause

\Wider[1cm]{
\begin{itemize}
\item Characteristic polynomials

\smallskip \pause

{\footnotesize
$\bullet$ Computation of the differential 

\smallskip \pause

$\bullet$ Definition of a polygon related to precision

\vspace{-1mm}

{\color{white} $\bullet$}
and comparison with the Hodge and the Newton polygon
}

\bigskip \pause

\item LU factorization

\smallskip \pause

{\footnotesize
$\bullet$ Computation of the differential \pause

$\bullet$ Heuristic estimation of the number of diffused digits
}

\bigskip \pause

\item Vector spaces and subspaces

\smallskip \pause

{\footnotesize
$\bullet$ Stable version of row elimination for computing

{\color{white} $\bullet$}
sums, intersections, direct images, inverse images

\smallskip \pause

$\bullet$ Computation of differentials (defined over subvarieties
of grassmannians) \pause

$\bullet$ Lattice {\sc vs} standard method for tracking precision
}
\end{itemize}}
\end{frame}

\begin{frame}
\frametitle{That's all folks}
\begin{center}
\Large Thank you for your attention!
\end{center}
\end{frame}


\end{document}
