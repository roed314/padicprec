\documentclass{beamer}
%\documentclass[handout]{beamer}

\usepackage[utf8]{inputenc}
\usepackage[english,francais]{babel}
\usepackage{amsmath,amssymb}
\usepackage{color}
\usepackage{tikz}
\usepackage{times}
\usepackage{verbatim}
\usepackage{array}
\usepackage{stmaryrd}
\usepackage{calc}

\newtheorem{prop}{Proposition}

\definecolor{bleu}{rgb}{0.8,0.8,0.6}
\definecolor{bleufonce}{rgb}{0,0,0.4}
\definecolor{rouge}{rgb}{0.7,0.2,0.2}
\definecolor{rougefonce}{rgb}{0.4,0,0}
\definecolor{vert}{rgb}{0.6,0.9,0.6}
\definecolor{vertfonce}{rgb}{0.1,0.4,0.1}
\definecolor{beige}{rgb}{0.9,0.9,0.7}
\definecolor{textcolor}{rgb}{0.8,0.8,0.6}
\definecolor{io}{rgb}{0.8,0.8,0.2}
\definecolor{comment}{rgb}{0.5,0.5,1}
\definecolor{command}{rgb}{0.3,1,0.3}

\newcommand\Wider[2][3.2em]{%
\makebox[\linewidth][c]{%
  \begin{minipage}{\dimexpr\textwidth+#1\relax}
  \raggedright#2
  \end{minipage}%
  }%
}

\newenvironment<>{varblock}[2][\textwidth]{
    \begin{center}
      \begin{minipage}{#1}
        \setlength{\textwidth}{#1}
        \setbeamercolor{block title}{fg=white,bg=jaune}
          \begin{actionenv}#3
            \def\insertblocktitle{#2}
            \par
            \usebeamertemplate{block begin}}
  {\par
      \usebeamertemplate{block end}
    \end{actionenv}
  \end{minipage}
\end{center}}

\newcommand{\ph}{\vphantom{A^A_A}}
\newcommand{\Z}{\mathbb Z}
\newcommand{\Zp}{\Z_p}
\newcommand{\Q}{\mathbb Q}
\newcommand{\Qp}{\Q_p}
\newcommand{\C}{\mathbb C}
\renewcommand{\Xi}{\mathcal C}

\newcommand{\E}{\mathbb E}
\newcommand{\val}{\text{\rm val}}

\setbeamercolor{background canvas}{bg=black}
\setbeamercolor{normal text}{fg=textcolor}
\setbeamercolor{structure}{fg=textcolor}
\setbeamercolor{frametitle}{bg=black,fg=white}
\setbeamertemplate{navigation symbols}{}%remove navigation symbols

\begin{document}

\title{Précision $p$-adique}
\author{Xavier Caruso}
\date{July 24, 2014}

\begin{frame}[plain]
\setbeamercolor{titre}{bg=black,fg=white}

\vspace{0.4cm}

\hfill
{\Large \bf \color{white} Précision $p$-adique}
\hfill \null

\vspace{0.5cm}

\hfill
{\footnotesize
%\begin{tabular}{>{\centering}p{3em}|>{\centering}p{4em}|>{\centering}p{3em}}
\begin{tabular}{>{\centering}p{3cm}>{\centering}p{3.5cm}>{\centering}p{3.5cm}l}
Xavier Caruso & David Roe & Tristan Vaccon & \\
Université Rennes 1 & Univ. British Columbia & 
Université Rennes 1 & \vspace{-0.1cm} \\
%\texttt
{\tiny xavier.caruso@normalesup.org} &
%\texttt
{\tiny roed.math@gmail.com} & 
%\texttt
{\tiny tristan.vaccon@univ-rennes1.fr} &
\end{tabular}
}
\hfill \null

\vspace{0.2cm}

{\color{bleu} \hfill \hrulefill \hfill \null}

\vspace{0.5cm}

\hfill
{\small \color{vert}
{\bf P}arameter spaces for 
{\bf E}fficient
{\bf A}rithmetic%
}
\hfill \null

\hfill
{\small \color{vert}
and {\bf C}urve security
{\bf E}valuation%
}
\hfill \null

\vspace{0.5cm}

\hfill
11 juin 2015
\hfill \null

\end{frame}

\begin{frame}
\begin{center}
\huge \color{white}
\bf Exemples
\end{center}
\end{frame}

\begin{frame}
\frametitle{Exemple 1 : la factorisation LU} \pause

On se donne $M \in M_n(\Z_p)$. \pause

\smallskip

On cherche à calculer des matrices :
$$L = 
\left( \!\!
\raisebox{-0.5\height + 0.2em}{
\begin{tikzpicture}[xscale=0.5,yscale=-0.5]
\node (A11) at (1,1) { $1$ };
\node (Ann) at (3.5,3.5) { $1$ };
\node (A12) at (2,1) { $0$ };
\node (A1n) at (3.5,1) { $0$ };
\node (Amn) at (3.5,2.5) { $0$ };
\node (A21) at (1,2) { $\star$ };
\node (An1) at (1,3.5) { $\star$ };
\node (Anm) at (2.5,3.5) { $\star$ };
\draw[dotted] (A11)--(Ann);
\draw[dotted] (A12)--(A1n)--(Amn)--(A12);
\draw[dotted] (A21)--(An1)--(Anm)--(A21);
\end{tikzpicture}}
\right)
\quad \text{et} \quad
U = 
\left( \!\!
\raisebox{-0.5\height + 0.2em}{
\begin{tikzpicture}[xscale=0.5,yscale=-0.5]
\node (A11) at (1,1) { $\star$ };
\node (Ann) at (3.5,3.5) { $\star$ };
\node (A1n) at (3.5,1) { $\star$ };
\node (A21) at (1,2) { $0$ };
\node (An1) at (1,3.5) { $0$ };
\node (Anm) at (2.5,3.5) { $0$ };
\draw[dotted] (A11)--(A1n)--(Ann)--(A11);
\draw[dotted] (A21)--(An1)--(Anm)--(A21);
\end{tikzpicture}}
\right)$$
telles que $M = LU$. 

\bigskip \pause

{\bf \color{white} Deux méthodes} \pause

\begin{itemize}
\item Pivot de Gauss

\smallskip 

\uncover<7->{
{\scriptsize
{\bf Avantage :} bonne complexité

\vspace{-0.1cm}

{\bf Inconvénient :} mauvaise stabilité numérique
}}

\pause

\item Formules à la Cramer

\smallskip 

\uncover<7->{
{\scriptsize
{\bf Avantage :} bonne stabilité numérique

\vspace{-0.1cm}

{\bf Inconvénient :} mauvaise complexité
}}
\end{itemize}

\end{frame}


\begin{frame}
\frametitle{Exemple 2 : les coefficients de Bézout} \pause

On se donne $A, B \in \Zp[X]$ unitaires de degré $n$.

\smallskip

On cherche à calculer des polynômes $U$ et $V$ tels que :

\vspace{-0.2cm}

$$A\:U + B\:V = 1$$

\smallskip \pause

{\bf \color{white} Deux méthodes}

\begin{itemize}
\item Algorithme d'Euclide

\smallskip

{\scriptsize
{\bf Avantage :} bonne complexité

\vspace{-0.1cm}

{\bf Inconvénient :} mauvaise stabilité numérique
}

\item Formules à la Cramer

\smallskip

{\scriptsize
{\bf Avantage :} bonne stabilité numérique

\vspace{-0.1cm}

{\bf Inconvénient :} mauvaise complexité
}
\end{itemize}

\medskip \pause

{\color{vert}
{\bf Objectif de cet exposé}

\smallskip \pause

Donner une méthode pour 
stabiliser les algorithmes rapides.}
\end{frame}

\begin{frame}
\begin{center}
\huge \color{white}
\bf Une théorie de 

la précision $p$-adique
\end{center}
\end{frame}

\begin{frame}
\frametitle{Suivi classique de la précision} \pause

\Wider{
\begin{tikzpicture}
\draw[transparent] (0,0.4) rectangle (12,8);

\begin{scope}[beige, rounded corners=10pt] 
\fill (0.1,1) rectangle (3.9,7.8);
\fill (4.1,1) rectangle (7.9,7.8);
\fill (8.1,1) rectangle (11.9,7.8);
\end{scope}

\begin{scope}[textcolor,thick]
\draw[->] (2.5,0.9) to[out=-25,in=-155] (5.5,0.9);
\node at (4,0.75) { $f_1$ };
\draw[->] (6.5,0.9) to[out=-25,in=-155] (9.5,0.9);
\node at (8,0.75) { $f_2$ };
\end{scope}

\only<3->{
  \draw[black,fill=green!50] (2,4.5) circle (0.8cm);
}
\only<4->{
\draw[black,fill=green!50,xshift=6cm,yshift=4.5cm,rotate=50] 
  (0,0) ellipse (0.8cm and 1.5cm);
\draw[black,fill=green!50,xshift=10cm,yshift=4.5cm,rotate=-20] 
  (0,0) ellipse (0.5cm and 2cm);
}

\only<5->{
  \draw[black,fill=blue!50,opacity=0.8] (6,4.5) circle (1.5cm);
}

\begin{scope}
\clip (8.1,1) rectangle (11.9,7.8);
\only<6->{
  \draw[black,fill=blue!50,opacity=0.8,xshift=10cm,yshift=4.5cm,rotate=20] 
    (0,0) ellipse (1.53cm and 3cm);
}
\only<7->{
  \draw[black,fill=red!50,opacity=0.8] (10,4.5) circle (3cm);
}
\end{scope}

\begin{scope}[textcolor!200,very thick, rounded corners=10pt] 
\draw (0.1,1) rectangle (3.9,7.8);
\draw (4.1,1) rectangle (7.9,7.8);
\draw (8.1,1) rectangle (11.9,7.8);
\end{scope}

\end{tikzpicture}
}
\end{frame}


\begin{frame}
\frametitle{Suivi de la précision pour les $p$-adiques paresseux} \pause

\Wider{
\begin{tikzpicture}
\draw[transparent] (0,0.4) rectangle (12,8);

\begin{scope}[beige, rounded corners=10pt] 
\fill (0.1,1) rectangle (3.9,7.8);
\fill (4.1,1) rectangle (7.9,7.8);
\fill (8.1,1) rectangle (11.9,7.8);
\end{scope}

\begin{scope}[thick]
\draw[->] (2.5,0.9) to[out=-25,in=-155] (5.5,0.9);
\node at (4,0.75) { $f_1$ };
\draw[->] (6.5,0.9) to[out=-25,in=-155] (9.5,0.9);
\node at (8,0.75) { $f_2$ };
\end{scope}

\only<4->{
  \draw[black,fill=green!50,xshift=6cm,yshift=4.5cm,rotate=50] 
    (0,0) ellipse (0.8cm and 1.5cm);
  \draw[black,fill=green!50,xshift=2cm,yshift=4.5cm,rotate=-20] 
    (0,0) ellipse (0.5cm and 2cm);
}
\only<3->{
  \draw[black,fill=green!50] (10,4.5) circle (0.8cm);
}

\only<5->{
  \draw[black,fill=blue!50,opacity=0.8] (6,4.5) circle (0.8cm);
}

\only<6->{
  \draw[black,fill=blue!50,opacity=0.8,xshift=2cm,yshift=4.5cm,rotate=20] 
    (0,0) ellipse (0.6cm and 0.3cm);
}
\only<7->{
  \draw[black,fill=red!50,opacity=0.8] (2,4.5) circle (0.3cm);
}

\begin{scope}[textcolor!200,very thick, rounded corners=10pt] 
\draw (0.1,1) rectangle (3.9,7.8);
\draw (4.1,1) rectangle (7.9,7.8);
\draw (8.1,1) rectangle (11.9,7.8);
\end{scope}

\end{tikzpicture}
}
\end{frame}

\begin{frame}
\frametitle{Boules et réseaux}

\begin{overlayarea}{\textwidth}{10.2cm}
\only<2->{
  On rappelle que $\Qp$ est muni de la norme $|x| = 
  p^{-\val(x)}$.
}

\smallskip

\only<3->{
On considère $E = \Qp^n$ muni de la norme infinie, notée $\Vert \cdot \Vert$ :

\smallskip

\hspace{0.5cm}
$\Vert (x_1, \ldots, x_n) \Vert = \max(|x_1|, \ldots, |x_n|).$
}

\medskip

\only<5->{
On pose 
$B(r) = \big\{ \,\, x \in E, \, \Vert x \Vert \leq r \,\, \big\}$
}

\smallskip

\only<8->{Ces boules sont stables par addition !}

\medskip

\only<10->{
Par définition, un \emph{réseau} $H$ de $E$ est

l'image de $B(1)$ par un isomorphisme

linéaire $\varphi : E \to E$.
}

\smallskip

\only<12->{
Les réseaux sont exactement les 

sous-$\Zp$-modules de $E$ engendrés

par une base de $E$ sur $\Qp$.
}

\smallskip

\only<13->{
{\bf Remarque importante :}

Les réseaux sont des objets exacts
}
\end{overlayarea}

\vspace{-9cm}

\hfill
\begin{tikzpicture}
\draw[transparent] (0,0) rectangle (4,6);

\only<4->{
\begin{scope}[beige, rounded corners=10pt] 
\fill (0.1,0.2) rectangle (3.9,5.8);
\end{scope}

\only<6>{
  \draw[black,fill=green!50] (2,3) circle (1cm);
  \fill[vertfonce] (2,3) circle (0.5mm);
  \node[vertfonce,scale=0.7] at (2.15,2.9) { $0$ };
  \node[vertfonce,scale=0.8] at (2,1.7) { $B(1)$ };
}
\only<7-8>{
  \draw[black,fill=green!50] (1.2,4.5) circle (1cm);
  \fill[vertfonce] (1.2,4.5) circle (0.5mm);
  \node[vertfonce,scale=0.7] at (1.35,4.4) { $x$ };
  \node[vertfonce,scale=0.8] at (1.2,3.2) { $x+B(1)$ };
}
\only<9-10>{
  \draw[black,fill=green!50] (2,3) circle (1cm);
  \fill[vertfonce] (2.5,3.5) circle (0.5mm);
  \node[vertfonce,scale=0.7] at (2.65,3.4) { $x$ };
  \node[vertfonce,scale=0.8] at (2,1.7) { $x+B(1) = B(1)$ };
}
\only<11->{
  \begin{scope}[xshift=2cm,yshift=3cm,rotate=20]
  \draw[black,fill=green!50] (0,0) ellipse (0.7cm and 1.5cm);
  \draw[black,->] (-1,0)--(1,0);
  \draw[black,->] (0,-2)--(0,2);
  \end{scope}
}
\begin{scope}[textcolor!200,very thick, rounded corners=10pt] 
\draw (0.1,0.2) rectangle (3.9,5.8);
\end{scope}}
\end{tikzpicture}
\end{frame}

\begin{frame}
\frametitle{Le lemme de précision} \pause

On se donne :

\medskip

\begin{tabular}{rl}
$E, F$ & deux $\Qp$-espaces vectoriels normés de dimension finie, \\
$U$ & un ouvert de $E$, \\
$f$ & une application différentiable de $U$ dans $F$.
\end{tabular}
\bigskip

\pause

Soit $x \in U$. 
\emph{On suppose que $f'(x) : E \to F$ est surjective.}

\medskip \pause

Alors
\uncover<5->{
$\forall \rho > 0,
\, \exists r_0 > 0, 
\, \forall r < r_0$ :}

pour tout réseau $H$
\only<4>{\og assez petit \fg}%
\only<5->{tel que $B^-(\rho r) \subset H \subset B(r)$ :}

\vspace{-0.2cm}

$$f(x+H) = f(x) + f'(x)(H).$$

\vspace{0.3cm} \pause \pause

{\bf Remarque} :

La dépendance de $r_0$ vis-à-vis de $\rho$ est explicite,

au moins si $f$ a plus de régularité.

\end{frame}


\begin{frame}
\frametitle{Le lemme de précision}

\Wider{
\begin{tikzpicture}
\draw[transparent] (0,0.4) rectangle (12,8);

\begin{scope}[beige, rounded corners=10pt] 
\fill (0.1,1) rectangle (3.9,7.8);
\fill (4.1,1) rectangle (7.9,7.8);
\fill (8.1,1) rectangle (11.9,7.8);
\end{scope}

\begin{scope}[thick]
\draw[->] (2.5,0.9) to[out=-25,in=-155] (5.5,0.9);
\node at (4,0.75) { $f_1$ };
\draw[->] (6.5,0.9) to[out=-25,in=-155] (9.5,0.9);
\node at (8,0.75) { $f_2$ };
\end{scope}

\draw[black,fill=green!50] (2,4.5) circle (0.8cm);
\draw[black,fill=green!50,xshift=6cm,yshift=4.5cm,rotate=50] 
  (0,0) ellipse (0.8cm and 1.5cm);
\draw[black,fill=green!50,xshift=10cm,yshift=4.5cm,rotate=-20] 
  (0,0) ellipse (0.5cm and 2cm);

\only<2->{
  \node[vertfonce,scale=0.8] at (2,3.4) { $x + H$ };
  \fill[vertfonce] (2,4.5) circle (0.5mm);
  \node[vertfonce,scale=0.7] at (2.15,4.4) { $x$ };
}
\only<3->{
  \node[vertfonce,scale=0.8] at (6,3) { $f_1(x) + f'_1(x)(H)$ };
  \fill[vertfonce] (6.5,4.4) circle (0.5mm);
  \node[vertfonce,scale=0.7] at (6.5,4.15) { $f_1(x)$ };
}
\only<4->{
  \node[vertfonce,scale=0.8] at (10,2.2) { $f(x) + f'(x)(H)$ };
  \node[vertfonce,scale=0.6] at (10,1.85) { avec $f = f_2 \circ f_1$ };
  \fill[vertfonce] (10.6,6) circle (0.5mm);
  \node[vertfonce,scale=0.7] at (10.5,5.75) { $f(x)$ };
}

\begin{scope}[textcolor!200,very thick, rounded corners=10pt] 
\draw (0.1,1) rectangle (3.9,7.8);
\draw (4.1,1) rectangle (7.9,7.8);
\draw (8.1,1) rectangle (11.9,7.8);
\end{scope}
\end{tikzpicture}
}
\end{frame}


\begin{frame}
\frametitle{Suivi adaptatif de la précision} \pause

\Wider{
\begin{tikzpicture}
\draw[transparent] (0,0.4) rectangle (12,8);

\begin{scope}[beige, rounded corners=10pt] 
\fill (0.1,1) rectangle (3.9,7.8);
\fill (4.1,1) rectangle (7.9,7.8);
\fill (8.1,1) rectangle (11.9,7.8);
\end{scope}

\begin{scope}[thick]
\draw[->] (2.5,0.9) to[out=-25,in=-155] (5.5,0.9);
\node at (4,0.75) { $f_1$ };
\draw[->] (6.5,0.9) to[out=-25,in=-155] (9.5,0.9);
\node at (8,0.75) { $f_2$ };
\end{scope}

\draw[black,fill=green!50] (2,4.5) circle (0.8cm);
\draw[black,fill=green!50,xshift=6cm,yshift=4.5cm,rotate=50] 
  (0,0) ellipse (0.8cm and 1.5cm);
\draw[black,fill=green!50,xshift=10cm,yshift=4.5cm,rotate=-20] 
  (0,0) ellipse (0.5cm and 2cm);

\node[vertfonce,scale=0.8] at (2,3.4) { $x + H$ };
\node[vertfonce,scale=0.8] at (6,3) { $f_1(x) + f'_1(x)(H)$ };
\node[vertfonce,scale=0.8] at (10,2.2) { $f(x) + f'(x)(H)$ };
\node[vertfonce,scale=0.6] at (10,1.85) { avec $f = f_2 \circ f_1$ };

\only<3-5>{
  \draw[black,fill=blue!50,opacity=0.8] (2.2,4.3) circle (0.4cm);
}
\only<4-5>{
  \draw[black,fill=blue!50,opacity=0.8,xshift=6.3cm,yshift=4.25cm,rotate=50] 
    (0,0) ellipse (0.4cm and 0.75cm);
}
\only<5>{
  \draw[black,fill=red!50,opacity=0.8]
    (6.3,4.25) ellipse (0.75cm);
}

\only<7-9>{
  \draw[black,fill=blue!50,opacity=0.8] (5.5,5) circle (0.4cm);
}
\only<8-9>{
  \draw[black,fill=blue!50,opacity=0.8,xshift=9.9cm,yshift=4.25cm,rotate=-30] 
    (0,0) ellipse (0.25cm and 0.45cm);
}
\only<9>{
  \draw[black,fill=red!50,opacity=0.8] (9.9,4.25) circle (0.45cm);
}
\only<10>{}
\begin{scope}[textcolor!200,very thick, rounded corners=10pt] 
\draw (0.1,1) rectangle (3.9,7.8);
\draw (4.1,1) rectangle (7.9,7.8);
\draw (8.1,1) rectangle (11.9,7.8);
\end{scope}
\end{tikzpicture}
}
\end{frame}

\begin{frame}
\frametitle{Suivi adaptatif de la précision}

{\bf \color{white} Avantages}

\medskip \pause

Les erreurs ne s'ajoutent \textbf{vraiment} pas

\smallskip \pause

{\scriptsize
Contrairement à ce qui se passe avec un suivi classique, 

\vspace{-0.15cm}

bien qu'on soit dans un cadre ultramétrique}

\medskip \pause

Concrètement, cette méthode de suivi permet dans 

les calculs de remplacer des \emph{sommes} par des \emph{max}

\bigskip \pause

{\bf \color{white} Une heuristique}

\medskip \pause

Sur des entrées aléatoires, un suivi adaptatif remplace un
comportement \emph{linéaire} en un comportement \emph{logarithmique}.

\medskip \pause

Si $x_1, \ldots, x_n$ des nombres $p$-adiques aléatoires, alors :

\begin{itemize}
\item
$\E \big[ \val(x_1) + \cdots + \val(x_n) \big] = \frac n{p-1}$
\item
$\E \big[ \max(\val(x_1), \ldots, \val(x_n)) \big] \sim \log_p n$
\end{itemize}

\end{frame}

\begin{frame}
\begin{center}
\huge \color{white}
\bf Retour à la

factorisation LU
\end{center}
\end{frame}

\begin{frame}
\frametitle{Découpage en étapes} \pause

\begin{tikzpicture}[xscale=2.2]
\node (M) at (0,0) { $M$ };
\node (LU) at (4,0) { $(L,U)$ };
\draw[|->] (M)--(LU) node[midway, above] { $f$ };

\uncover<3->{
\node (LU0) at (0,-1) { $(L_0,U_0)$ };
\node (LU1) at (1.1,-1) { $(L_1,U_1)$ };
\node (LU2) at (2.2,-1) { $(L_2,U_2)$ };
\node (LUi) at (3.1,-1) { $\cdots$ };
\node (LUn) at (4,-1) { $(L_n,U_n)$ };
\draw[transform canvas={xshift=0.6mm}] (M)--(LU0);
\draw[transform canvas={xshift=-0.6mm}] (M)--(LU0);
\draw[transform canvas={xshift=0.6mm}] (LU)--(LUn);
\draw[transform canvas={xshift=-0.6mm}] (LU)--(LUn);
\begin{scope}[|->]
\draw (LU0)--(LU1) node[midway, above] { $f_1$ };
\draw (LU1)--(LU2) node[midway, above] { $f_2$ };
\draw (LU2)--(LUi) node[midway, above] { $f_3$ };
\draw (LUi)--(LUn) node[midway, above] { $f_n$ };
\end{scope}}
\end{tikzpicture}

\medskip \pause \pause

où $(L_i, U_i)$ est défini par $L_i U_i = M$ et :
$$L_i = 
\left( \!\!
\raisebox{-0.5\height + 0.2em}{
\begin{tikzpicture}[xscale=0.5,yscale=-0.5]
\draw[vertfonce](4,0.5)--(4,7.5);
\draw[vertfonce](0.5,4)--(7.5,4);
\node (A11) at (1,1) { $1$ };
\node (Aii) at (3.5,3.5) { $1$ };
\node (A12) at (2,1) { $0$ };
\node (A1n) at (7,1) { $0$ };
\node (Aij) at (4.5,3.5) { $0$ };
\node (Ain) at (7,3.5) { $0$ };
\node (A21) at (1,2) { $\star$ };
\node (An1) at (1,7) { $\star$ };
\node (Aji) at (3.5,4.5) { $\star$ };
\node (Ajn) at (7,4.5) { $\star$ };
\node (Ann) at (7,7) { $\star$ };
\draw[dotted] (A11)--(Aii);
\draw[dotted] (A12)--(A1n)--(Ain)--(Aij)--(A12);
\draw[dotted] (A21)--(An1)--(Ann)--(Ajn)--(Aji)--(A21);
\end{tikzpicture}}
\right),
\quad
U_i = 
\left( \!\!
\raisebox{-0.5\height + 0.2em}{
\begin{tikzpicture}[xscale=0.5,yscale=-0.5]
\draw[vertfonce](4,0.5)--(4,7.5);
\draw[vertfonce](0.5,4)--(7.5,4);
\node (A11) at (1,1) { $\star$ };
\node (A1n) at (7,1) { $\star$ };
\node (Aii) at (3.5,3.5) { $\star$ };
\node (Ain) at (7,3.5) { $\star$ };
\node (A21) at (1,2) { $0$ };
\node (An1) at (1,7) { $0$ };
\node (Anm) at (6,7) { $0$ };
\node (Ajj) at (4.5,4.5) { $1$ };
\node (Ann) at (7,7) { $1$ };
\node (Ajk) at (5.5,4.5) { $0$ };
\node (Ajn) at (7,4.5) { $0$ };
\node (Amn) at (7,6) { $0$ };
\draw[dotted] (Ajj)--(Ann);
\draw[dotted] (A11)--(A1n)--(Ain)--(Aii)--(A11);
\draw[dotted] (A21)--(An1)--(Anm)--(A21);
\draw[dotted] (Ajk)--(Ajn)--(Amn)--(Ajk);
\end{tikzpicture}}
\right)$$

\pause

Autrement dit, $(L_i, U_i)$ est le couple obtenu par l'algorithme de 
Gauss après réduction de la $i$-ième ligne.

\medskip \null
\end{frame}

\begin{frame}
\frametitle{Calcul différentiel} \pause

On pose 
$g_i = f_i \circ f_{i-1} \circ \cdots \circ f_1 \pause 
: M \mapsto (L_i, U_i)$.

\medskip \pause

{\bf \color{white}
Le calcul de $g_i'(M)$}

\vspace{-0.7cm} \pause

\begin{align*}
M & = L_i \cdot U_i \bigskip \\
\uncover<6->{dM & = L_i \cdot d U_i + d L_i \cdot U_i \\}
\uncover<7->{%
L_i^{-1} \cdot dM \cdot U_i^{-1} & = 
{\only<7>{\color{transparent}}
\underbrace{\color{textcolor}d U_i \cdot U_i^{-1}}_{%
\left( \!\!
\raisebox{-0.5\height + 0.2em}{
\begin{tikzpicture}[xscale=0.2,yscale=-0.2]
\node[scale=0.4] (A11) at (1,1) { $\star$ };
\node[scale=0.4] (A1n) at (6,1) { $\star$ };
\node[scale=0.4] (Aii) at (3.5,3.5) { $\star$ };
\node[scale=0.4] (Ain) at (6,3.5) { $\star$ };
\node[scale=0.4,transparent] (Ann) at (6,6) { $\star$ };
\draw[dotted] (A11)--(A1n)--(Ain)--(Aii)--(A11);
\end{tikzpicture}}
\right)}} + 
{\only<7-8>{\color{transparent}}
\underbrace{\color{textcolor}L_i^{-1} \cdot d L_i}_{%
\left( \!\!
\raisebox{-0.5\height + 0.2em}{
\begin{tikzpicture}[xscale=0.2,yscale=-0.2]
\node[scale=0.4,transparent] (A11) at (1,1) { $\star$ };
\node[scale=0.4] (A21) at (1,2) { $\star$ };
\node[scale=0.4] (An1) at (1,6) { $\star$ };
\node[scale=0.4] (Aji) at (3.5,4.5) { $\star$ };
\node[scale=0.4] (Ajn) at (6,4.5) { $\star$ };
\node[scale=0.4] (Ann) at (6,6) { $\star$ };
\draw[dotted] (A21)--(An1)--(Ann)--(Ajn)--(Aji)--(A21);
\end{tikzpicture}}
\right)}}}
\end{align*}

\pause \pause \pause \pause \pause

{\bf \color{white}
Conséquences} \pause

\smallskip

En notant $\rho_i$ la norme du $i$-ième mineur principal de $M$, on a

\begin{itemize}
\item $\Vert g_i'(M) \Vert \leq \max(1, \rho_1, \ldots, \rho_i)^3$ \smallskip
\item $\Vert g_i'(M)^{-1} \Vert \leq \max(1, \rho_1, \ldots, \rho_i)$
\end{itemize}

\vspace{-1.7cm} \pause

{\color{red!50} 
\null \hspace{6.8cm}
{\bf NB :} On a un \emph{max},

\vspace{-0.5cm}

\null \hspace{6.8cm}
\phantom{{\bf NB :} }pas un produit}

\vspace{0.5cm} \null
\end{frame}

\begin{frame}
\frametitle{La méthode du suivi adaptatif} \pause

$\rho_i = | \det M_{\{1,\ldots,i\}, \{1, \ldots, i\}} |$
\hfill ; \hfill $r_i = \max(1, \rho_1, \ldots, \rho_i)$ \pause
\hfill ; \hfill $H = M_n(\Z_p)$

\bigskip \pause

\Wider{
\begin{tikzpicture}
\draw[transparent] (0,1) rectangle (12,7.8);

\begin{scope}[beige, rounded corners=10pt] 
\fill (0.5,1) rectangle (5.5,7.8);
\fill (6.5,1) rectangle (11.5,7.8);
\end{scope}

\draw[thick,->] (5.6,4.4)--(6.4,4.4)
  node[midway,above] { $f_i$ };

\draw[black,fill=green!50,xshift=3cm,yshift=4.4cm,rotate=-70] 
  (0,0) ellipse (0.7cm and 2.2cm);
\draw[black,fill=green!50,xshift=9cm,yshift=4.4cm,rotate=50] 
  (0,0) ellipse (0.8cm and 1.5cm);

\node[vertfonce,scale=0.8] at (3,7.2) { $g_{i-1}(M) + g'_{i-1}(M)(H)$ };
\node[vertfonce,scale=0.8] at (9,7.2) { $g_i(M) + g'_i(M)(H)$ };

\only<5->{
  \draw[bleufonce] (3,4.4) circle (0.6cm);
  \draw[bleufonce,<->] (3,4.4)--(3,5) node[midway,right,scale=0.5] { $r_{i-1}^{-1}$ };
}
\only<6->{
  \draw[rougefonce] (3,4.4) circle (2.3cm);
  \draw[rougefonce,<->] (3,4.4)--(1.16,3.02) node[midway,above left,scale=0.5] { $r_{i-1}^3$ };
}
\only<7->{
  \draw[bleufonce] (9,4.4) circle (0.6cm);
  \draw[bleufonce,<->] (9,4.4)--(9,5) node[midway,right,scale=0.5] { $r_i^{-1}$ };
  \draw[rougefonce] (9,4.4) circle (2.3cm);
  \draw[rougefonce,<->] (9,4.4)--(7.16,3.02) node[midway,above left,scale=0.5] { $r_i^3$ };
}
\only<8-9>{
  \draw[black,fill=blue!50,opacity=0.8] (3.2,4.2) circle (0.2cm);
}
\only<9>{
  \draw[black,fill=blue!50,opacity=0.8] (9.1,4.4) circle (0.45cm);
}

\only<-10>{
  \begin{scope}[textcolor!200,very thick, rounded corners=10pt] 
  \draw (0.5,1) rectangle (5.5,7.8);
  \draw (6.5,1) rectangle (11.5,7.8);
  \end{scope}
}
\end{tikzpicture}
}
\end{frame}

\begin{frame}
\frametitle{Schéma de l'algorithme stabilisé} \pause

{\color{black!20}
\makebox[1.5cm][l]{\bf Entrée :} 
une matrice $M \in M_n(\Zp)$ connu à précision $O(p^N)$

\smallskip

\makebox[1.5cm][l]{\bf Sortie :} 
la décomposition LU de $M$
}

\bigskip \pause

$L_0 \leftarrow M$; \, $U_0 \leftarrow I_n$

\smallskip \pause

$v_0 \leftarrow 0$ \pause
\hfill {\color{comment} // $v_i = - \log_p r_i$
avec les notations précédentes}

\smallskip \pause

{\color{rouge} pour} $i = 1, 2, \ldots, n$ {\color{rouge} :}

\smallskip \pause

\hspace{0.6cm}
{\color{comment}
// On calcule $(L_i, U_i)$ à partir $(L_{i-1}, U_{i-1})$}

\smallskip \pause

\hspace{0.6cm}
$\mu_i \leftarrow \val((L_{i-1})_{i,i})$

\hspace{0.6cm}
$\nu_i \leftarrow \nu_{i-1} + \mu_i$ \pause
\hfill {\color{comment} 
// $\nu_i = \val(\det M_{\{1,\ldots,i\}, \{1,\ldots,i\}}) = - \log_p \rho_i$}

\pause

\hspace{0.6cm}
$v_i \leftarrow \min(v_{i-1}, \nu_i)$

\smallskip \pause

\hspace{0.6cm}
{\color{command} relever} 
$(L_{i-1}, U_{i-1})$ à précision $O(p^{N-3 v_i - 2 \min(0,\mu_i)})$

\smallskip \pause

\hspace{0.6cm}
{\color{command} calculer}
$(L_i, U_i)$ par l'algorithme de Gauss \pause

\hfill {\color{comment} // $(L_i, U_i)$ est connu au moins à précision
$O(p^{N-3 v_i})$}

\medskip \pause

{\color{command} retourner}
$(L_n, U_n)$ à précision $O(p^{N+v_n})$
\end{frame}

\begin{frame}
\frametitle{That's all folks}


\Wider{
\begin{tikzpicture}
\begin{scope}[beige, rounded corners=10pt] 
\fill (0.5,1) rectangle (5.5,7.8);
\fill (6.5,1) rectangle (11.5,7.8);
\end{scope}

\draw[thick,->] (5.6,4.4)--(6.4,4.4)
  node[midway,above] { $f_i$ };

\draw[black,fill=green!50,xshift=3cm,yshift=4.4cm,rotate=-70] 
  (0,0) ellipse (0.7cm and 2.2cm);
\draw[black,fill=green!50,xshift=9cm,yshift=4.4cm,rotate=50] 
  (0,0) ellipse (0.8cm and 1.5cm);

\node[vertfonce,scale=0.8] at (3,7.2) { $g_{i-1}(M) + g'_{i-1}(M)(H)$ };
\node[vertfonce,scale=0.8] at (9,7.2) { $g_i(M) + g'_i(M)(H)$ };

  \draw[bleufonce] (3,4.4) circle (0.6cm);
  \draw[bleufonce,<->] (3,4.4)--(3,5) node[midway,right,scale=0.5] { $r_{i-1}^{-1}$ };
  \draw[rougefonce] (3,4.4) circle (2.3cm);
  \draw[rougefonce,<->] (3,4.4)--(1.16,3.02) node[midway,above left,scale=0.5] { $r_{i-1}^3$ };
  \draw[bleufonce] (9,4.4) circle (0.6cm);
  \draw[bleufonce,<->] (9,4.4)--(9,5) node[midway,right,scale=0.5] { $r_i^{-1}$ };
  \draw[rougefonce] (9,4.4) circle (2.3cm);
  \draw[rougefonce,<->] (9,4.4)--(7.16,3.02) node[midway,above left,scale=0.5] { $r_i^3$ };
  \draw[black,fill=blue!50,opacity=0.8] (3.2,4.2) circle (0.2cm);
  \draw[black,fill=blue!50,opacity=0.8,xshift=9.1cm, yshift=4.4cm, rotate=15]
    (0,0) ellipse (0.4cm and 0.5cm);

  \begin{scope}[textcolor!200,very thick, rounded corners=10pt] 
  \draw (0.5,1) rectangle (5.5,7.8);
  \draw (6.5,1) rectangle (11.5,7.8);
  \end{scope}

\fill[black, opacity=0.8] (0,0.5) rectangle (12,8);

\begin{scope}[xshift=6cm, yshift=4cm, rotate=5]
  \fill[green!50] (0,0) ellipse (5cm and 2cm);
  \fill[blue!50, opacity=0.8] (1,-1) circle (0.5cm);
  \fill[blue!50, opacity=0.8] (-2,-0.5) circle (0.7cm);
  \fill[blue!50, opacity=0.8] (0.5,1.2) circle (0.5cm);
  \fill[blue!50, opacity=0.8] (2,0.1) circle (0.2cm);
  \fill[blue!50, opacity=0.8] (3.5,-0.2) circle (0.15cm);
  \node[vertfonce,rotate=5,xscale=1.8,yscale=2.5] at (0,0) { \bf Merci de votre attention ! };
\end{scope}
\end{tikzpicture}}
\end{frame}

\end{document}
